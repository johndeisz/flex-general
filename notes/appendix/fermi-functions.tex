\chapter{Analytic Fermi functions}
\label{chapter:fermi-functions}

\section{Fermi frequency sums}

In the vicinity of $z = \pi T i + \delta z$, we have
\begin{eqnarray}
\frac{1}{e^{z \beta} + 1} & = & \frac{1}{e^{(\pi T i + \delta z)} + 1} \\
\label{form-one}
& = & \frac{1}{-e^{\delta z\, \beta} + 1} \\
& \simeq & \frac{-1}{\delta z \beta}. 
\end{eqnarray}
Consequently,
\begin{equation}
\frac{1}{2 \pi i} \oint_1 dz \frac{F(z)}{e^{z \beta} + 1}
= -T \sum_{\varepsilon_n} F(i \varepsilon_n)
\end{equation}
where $\varepsilon_n = (2n+1)\pi T, n= \cdots -2,-1,0,1,2 \cdots$
are the simple poles of $1/(\exp(z \beta) + 1)$ and
the contour encloses the imaginary axis in the
standard (CCW) direction.
Thus,
\begin{eqnarray}
T \sum_{\varepsilon_n} F(i \varepsilon_n) & = &
 \frac{-1}{2 \pi i} \oint_1 dz \frac{F(z)}{e^{z \beta}+1} \\
& = & \frac{1}{2 \pi i} \oint_2 dz \frac{F(z)}{e^{z \beta}+1} \\
& = & \mathrm{Res}\; \frac{F(z)}{e^{z \beta + 1}}.
\end{eqnarray}
Here we have distorted the contour 
($1 \to 2$) into left and right semicircles
that enclose everything but the imaginary axis.  
The residue corresponds to any pole such that Re~$z \ne 0$.

The distortion of the contour is allowed when $F(z)$ is such
that the integrand vanishes at least as $1/z^{1 + \eta}$
for $|z| \to \infty$.  In certain cases, this will
not occur, but we may be able to change the integrand
to make this occur.  Namely
\begin{eqnarray}
T \sum_{\varepsilon} F(i \varepsilon_n) & = &
\frac{1}{2 \pi i} \oint_1 dz \frac{F(z)}{e^{-z \beta}+1} \\
\label{form-two}
& = & - \frac{1}{2 \pi i} \oint_2 dz \frac{F(z)}{e^{-z \beta}+1} \\
& = & - \mathrm{Res}\, \frac{F(z)}{e^{-z \beta} + 1}
\end{eqnarray} 

\section{$Q$-functions}

In frequency space, the analytic $Q$ functions take the
form
\begin{eqnarray}
Q_{0j}(\varepsilon_n) & = & \frac{1}{2}
\left[ \frac{1}{i\varepsilon_n - x_{0j}}
 + \frac{1}{i\varepsilon_n + x_{0j}} \right] \\
Q_{1j}(\varepsilon_n) & = & 
\frac{1}{2x_{1j}}\left[ \frac{1}{i\varepsilon_n - x_{1j}}
 - \frac{1}{i\varepsilon_n + x_{1j}} \right]
\end{eqnarray}
These functions have $1 /(i \varepsilon_n)$ and
$1/(i\varepsilon_n)^2$ asymptotic frequency dependencies respectively.

To transform to $\tau$ space we use
\begin{equation}
Q_{ij}(\tau) \equiv T \sum_{\varepsilon_n} e^{-i \varepsilon_n \tau} 
Q_{ij}(\varepsilon_n)
\end{equation}
If $\tau > 0$, we need to use Eq.~(\ref{form-two}) so that
the contour deformation is valid.
Doing gives the following formulas for $Q_{ij}(\tau)$:
\begin{eqnarray}
Q_{0j}(\tau > 0) & = & -\frac{1}{2} \left[
\frac{e^{-x_{0j} \tau}}
{e^{-x_{0j} \beta} + 1} + \frac{e^{x_{0j} \tau}}{e^{x_{0j} \beta} + 1} 
\right] \\
Q_{1j}(\tau > 0) & = & -\frac{1}{2 x_{1j}}
\left [\frac{e^{-x_{1j} \tau}}
{e^{-x_{1j} \beta} + 1} -  \frac{e^{x_{1j} \tau}}
{e^{x_{1j} \beta} + 1} \right]
\end{eqnarray}
For $\tau < 0$ we use Eq.~(\ref{form-one}) to get
\begin{eqnarray}
Q_{0j}(\tau < 0) & = & \frac{1}{2} \left[
\frac{e^{-x_{0j} \tau}}{e^{x_{0j} \beta} + 1} +
\frac{e^{x_{0j} \tau}}{e^{-x_{0j} \beta} + 1} \right] \\
Q_{1j}(\tau < 0) & = & \frac{1}{2x_{1j}} \left[
\frac{e^{-x_{1j} \tau}}{e^{x_{1j} \beta} + 1} -
\frac{e^{x_{1j} \tau}}{e^{-x_{1j} \beta} + 1} \right] 
\end{eqnarray}

In general, we can write
\begin{eqnarray}
Q_{ij}(\tau > 0) & = &
a^+_{0 ij} \frac{e^{x_{ij} \tau}}
{e^{x_{ij}\beta} + 1} + 
a^+_{1 ij} \frac{e^{-x_{ij} \tau}}
{e^{-x_{ij}\beta} + 1} \\
Q_{ij}(\tau < 0) & = &
a^-_{0 ij} \frac{e^{-x_{ij} \tau}}
{e^{x_{ij}\beta} + 1} + 
a^-_{1 ij} \frac{e^{x_{ij} \tau}}
{e^{-x_{ij}\beta} + 1} 
\end{eqnarray}
where
\begin{eqnarray}
a^+_{00j} & = & -\frac{1}{2} \\
a^+_{10j} & = & -\frac{1}{2} \\
a^+_{01j} & = & \frac{1}{2 x_{1j}} \\
a^+_{11j} & = & -\frac{1}{2 x_{1j}} \\
a^-_{i0j} & = & -a^{+}_{i0j} \\
a^-_{i1j} & = & a^{+}_{i1j}
\end{eqnarray}
This formulation will prove helpful in evaluating
integrals that include more than one analytic function.





