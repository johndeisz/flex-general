\chapter{Analytic Bose functions}
\label{chapter:bose-functions}

\section{Bose frequency sums}
Consider a Bose Green's function with a simple pole,
\begin{equation}
G(i \omega_n) =  \frac{1}{i \omega_n - y},
\end{equation}
and its Fourier transform to the $\tau$-domain,
\begin{equation}
G(\tau)  =  T \sum_n e^{-i\omega_n \tau} G(\omega_n).
\end{equation}
To proceed towards evaluation of the frequency sum, consider
$(\exp(\beta z) - 1)^{-1}$ in the limit
$z \to 0$:
\begin{equation}
(e^{\beta z} - 1)^{-1} \simeq (\beta z)^{-1}.
\end{equation}
By the residue theorem we have
\begin{equation}
\frac{1}{2 \pi i} \oint_{|z|=\delta}\; dz \frac{F(z)}{z} 
 = F(z = 0).
\end{equation}
This can be rewitten as
\begin{equation}
\frac{1}{2 \pi i} \oint dz \; \frac{\beta}{e^{\beta z}-1} 
F(z) = F(z = 0)
\end{equation}
or, finally, 
\begin{equation}
\frac{1}{2 \pi i} \oint dz \; \frac{1}{e^{\beta z}-1} 
F(z) = T \, F(z = 0)
\end{equation}
If the contour encloses all the poles of
$(\exp(\beta z)-1)^{-1}$ at $z = i\omega_n$
where $\omega_n = 2 n \pi T$, then
\begin{equation}
\frac{1}{2 \pi i} \oint dz \; \frac{1}{e^{\beta z}-1} 
F(z) = T \sum_n F(i\omega_n)
\end{equation}
Assuming that $F(z) / (\exp(\beta z) - 1)$
decreases \textit{at least} like $|z|^{-1 - \epsilon}$
as $|z| \to \infty$, then we can redefine the
contour enclosing the  complex axis (\textit{i.e.} Re $z$ = 0)
to two semi-circular contours covering the complex plane
except for Re $z$ = 0.

With this we have
\begin{equation}
T \sum_n F(i\omega_n) = -\mathrm{Res}
\left( \frac{F(z)}{e^{\beta z}-1}\right)
\end{equation}
with the minus sign resulting from the contour integrations being
taken in the clockwise directions.

By the same reasoning we can show that
\begin{equation}
\frac{-1}{2 \pi i} \oint dz \frac{F(z)}{e^{-\beta z}-1}
= T \sum_n F(i \omega_n)
\end{equation}
so that
\begin{equation}
T \sum_n F(i\omega_n) = \mathrm{Res}\left(\frac{F(z)}
{e^{-\beta z}-1}\right).
\end{equation}

We note that in these derivations we have assumed that
there are no poles in $F(z)$ for $\mathrm{Re}\;z = 0$.
If there is such a pole, the above proofs must be modified and
an extra term will result.

\section{Fourier transforms of simple-pole functions}

Consider
\begin{eqnarray}
G(\tau > 0) & = & T \sum_n e^{-i \omega_n \tau} \frac{1}{i \omega_n - y} \\
& = & \frac{-1}{2 \pi i} \oint dz\;\frac{e^{-z \tau}}{(e^{-z \beta} - 1)
(z - y)} \\
& = & \frac{e^{-y \tau}}{e^{-y \beta} - 1}  \\
& = & \frac{-1}{1 - e^{-y \beta}} e^{-y \tau} \\
& = & - \frac{e^{y\beta}}{e^{y\beta} - 1} e^{- y \tau} \\
& = & - (1 + n_y) e^{-y \tau} 
\end{eqnarray}

For $\tau < 0$
\begin{eqnarray}
G(\tau) & = & T \sum_n \frac{e^{-i \omega_n \tau}}{i\omega_n - y} \\
& = & \frac{1}{2 \pi i} \oint dz\; \frac{e^{-z \tau}}{(e^{\beta z} - 1)
(z - y)} \\
& = & - \frac{e^{-y \tau}}{e^{\beta y} - 1} \\
& = & - n_y e^{-y \tau}
\end{eqnarray}

These results agree with AGD.  Also, we have
\begin{eqnarray}
G(\tau \to 0^+) - G(\tau \to 0^-) & = & - (1 + n_y) - (- n_y) \\
& = & -1
\end{eqnarray}
as required.

\section{$R$-functions}

We define $R_o$ in frequency space where it has the form
\begin{equation}
R_o(i \omega_m, y_o) = \frac{1}{2} \left[ \frac{1}{i \omega_m - y_o}
             + \frac{1}{i \omega_m + y_o} \right] 
\end{equation}
Using the results from the previous sections, we obtain
the $\tau$-dependence of this function:
\begin{equation}
R_o(\tau > 0, y_o)  =  \frac{-1}{2}( 1 - n_{y_o})e^{-y_o \tau} +
 \frac{-1}{2}( 1 - n_{-y_o})e^{y_o \tau}
\end{equation}	
Note that $n_{y_o}$ may be negative.  This  does not pose any
difficult since in this case  $n_{y_o}$ does not represent
the number density of a real particle; it is simply a mathematical
function that is used in various manipulations.
Simplifying the above we get
\begin{eqnarray}
R_o(\tau > 0, y_o) & = & \frac{-1}{2} \left[
  \frac{e^{\beta y_o}}{e^{\beta y_o} - 1} e^{-y_o \tau}
+ \frac{e^{-\beta y_o}}{e^{-\beta y_o} -1} e^{y_o \tau} \right] \\
& = & \frac{-1}{2}\left[ \frac{e^{y_o(\beta/2 - \tau)}}
{e^{\beta y_o / 2} - e^{-\beta y_o / 2}} +
\frac{e^{-y_o(\beta /2 - \tau)}}
{e^{-\beta y_o /2} - e^{\beta y_o / 2}} \right] \\
& = & \frac{-1}{2}\left[ \frac{ e^{y_o(\beta/2 - \tau)} -
e^{-y_o(\beta_2 - \tau)}}{e^{\beta y_o /2} - e^{-\beta y_o / 2}} \right] \\
& = & \frac{-1}{2} \frac{\sinh(y_o (\beta/2 - \tau))}
{\sinh(y_o \beta / 2)}
\end{eqnarray}
Also, we have
\begin{eqnarray}
R_o(\tau < 0) & = & R_o(\tau + \beta > 0) \\
& = & \frac{-1}{2} \frac{\sinh(y_o (\beta / 2 - (\tau + \beta)))}
{\sinh(y_o \beta / 2)} \\
& = & \frac{-1}{2} \frac{\sinh(y_o(-\beta/2 - \tau))}
{\sinh(y_o \beta / 2)} \\
& = & \frac{1}{2} \frac{\sinh(y_o(\beta /2 + \tau))}
{\sinh(y_o \beta / 2)}
\end{eqnarray}

The second analytic function, $R_1$, which has an
$1 / (\omega_m)^2$ asymptotic frequency dependence, is
given by
\begin{equation}
R_1(i\omega_m, y_1) = \frac{1}{2 y_1}
\left[ \frac{1}{i \omega_m - y_1} - \frac{1}{i\omega_m + y_1} \right].
\end{equation}
In $\tau$-space we have
\begin{eqnarray}
R_1(\tau > 0, y_1) & = & 
\frac{-1}{2 y_1} \left[ (1 + n_{y_1})e^{-y_1 \tau} -
(1 + n_{-y_1})e^{y_1 \tau} \right] \\
& = & \frac{-1}{2 y_1} 
\left[ \frac{e^{y_1(\beta/2 - \tau)} + e^{-y_1(\beta/2 - \tau)}}
{e^{y_1 \beta /2 } - e^{y_1 \beta / 2}} \right] \\
& = & \frac{-1}{2 y_1} \frac{\cosh(y_1(\beta/2 - \tau))}
{\sinh(y_1 \beta / 2)}
\end{eqnarray}
For $\tau < 0$ we have
\begin{eqnarray}
R_1(\tau < 0, y_1) & = & R_1(\tau + \beta > 0) \\
& = & \frac{-1}{2 y_1} \frac{\cosh(y_1(-beta/2 - \tau))}
{\sinh(y_1 \beta / 2)} \\
& = & \frac{-1}{2 y_1} \frac{\cosh(y_1(\beta/2 + \tau))}
{\sinh(y_1 \beta /2)}
\end{eqnarray} 

Finally, we also include some alternate expressions for $R_0$
and $R_1$ that only depend on exponentials having negative
arguments.
\begin{eqnarray}
R_o(\tau > 0, y_o) & = & \frac{-1}{2}
\left[ \frac{e^{y_o \beta}}{e^{y_o \beta} - 1} e^{-y_o \tau}
+ \frac{e^{-y_o \beta}}{e^{-y_o \beta} - 1} e^{y_o \tau} \right] \\
& = & \frac{-1}{2} \left[ \frac{e^{-y_o \tau}}{1 - e^{-y_o \beta}}
+ \frac{e^{-y_o(\beta - \tau)}}{e^{-y_o \beta} - 1} \right] \\
& = & \frac{-1}{2} \left[ \frac{e^{-y_o \tau} -
e^{-y_o(\beta - \tau)}}{1 - e^{-y_o \beta}} \right ]
\end{eqnarray}
and for $R_1$
\begin{eqnarray}
R_1(\tau >0, y_1) & = & \frac{-1}{2 y_1}
\left[ \frac{e^{y_1 \beta}}{e^{y_1 \beta} - 1} e^{-y_1 \tau}
- \frac{e^{-y_1 \beta}}{e^{-y_1 \beta} - 1} e^{y_1 \tau} \right] \\
& = & \frac{-1}{2 y_1} 
\left[ \frac{e^{-y_1 \tau}}{1 - e^{-y_1 \beta}} -
\frac{e^{-y_1(\beta - \tau)}}{e^{-y_1 \beta} - 1} \right] \\
& = & \frac{-1}{2 y_1} \left[ \frac{e^{-y_1 \tau} +
e^{-y_1(\beta - \tau)}}{1 - e^{-\beta y_1}} \right]
\end{eqnarray}





