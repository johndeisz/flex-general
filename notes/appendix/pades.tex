\chapter{Pade approximants}

Consider a function $A(z)$ whose argument (frequency), 
$z$, is complex. 
If the high frequency behavior goes as $c/ z$
where $c$ is a constant,
then a possible fitting function for $A$ would be
\begin{equation}
\label{pade}
A_{pade}(z) = \frac{p_0 + p_1 z +  ...  + p_n z^n}
{  q_0 + q_1 z + ... + q_n z^n + z^{n+1}}
\end{equation}
where $\{p_i, q_i\}$ provide $2n + 2$ fitting constants
and we will call $n$ the order of this Pade approximant.
As $|z| \to \infty$, $A_{pade}(z) \to p_n / z$ as required.

The function $A(z)$ is, generally, non-analytic across
the real axis ($\textrm{Im } z = 0$).  Thus, if we wish to
obtain a fit to $A(z)$ that is valid just above the real axis,
\textit{i.e.} $z = \varepsilon + i\delta$, then we can
only use data for $A(z)$ whose $z$ values have $\textrm{Im }z > 0$.
In our case, we have data for $A(z)$ on the imaginary frequency axis,
\textit{i.e.} $z = i \varepsilon_j$ where $\varepsilon_j = 
(2 j + 1)\pi T$ where $j = 0, 1, 2, ...$ and $T$ is the temperature.
To obtain a unique fit, we need $2n + 2$ constraints.  We get
these by using results for
$A(z = i\varepsilon_j)$ at the $2n + 2$
lowest positive values of $\varepsilon_j$.

This gives use a set of $2n +2$ linear equations for solving
the unknown constants $\{p_i, q_i\}$.  For example, for the
first frequency value, $\varepsilon_0 = \pi T$, we have
\begin{equation}
A(i\varepsilon_0)  =  
\frac{p_0 + p_1 (i\varepsilon_0) + ... + p_n (i\varepsilon_0)^n}
{q_0 + q_1 (i\varepsilon_0) + ... + q_n (i\varepsilon_0)^n +
(i\varepsilon_0)^{n+1} }. 
\end{equation}
After multiplying the equation by the denominator of the
right-hand side and doing some elementary algebra we get
\begin{equation}
A(i\varepsilon_0)\,q_0 +
A(i\varepsilon_0)\,(i\varepsilon_0) q_1 + ... +
A(i\varepsilon_0)^n\,(i\varepsilon_0)^n q_n -
p_0 - 
(i\varepsilon_0) p_1 - ... -(i\varepsilon_0)^n p_n 
= -A(i\varepsilon_0) (i\varepsilon_0)^{n+1}
\end{equation}
Likewise, for the next frequency value, $\varepsilon_1 = 3 \pi T$
we have
\begin{equation}
A(i\varepsilon_1)\,q_0 +
A(i\varepsilon_1)\,(i\varepsilon_1) q_1 + ... +
A(i\varepsilon_1)^n\,(i\varepsilon_1)^n q_n -
p_0 - 
(i\varepsilon_1) p_1 - ... -(i\varepsilon_1)^n p_n 
= -A(i\varepsilon_1) (i\varepsilon_1)^{n+1}
\end{equation}

This set of $2n + 2$ equations in $2n + 2$ unknowns can be
written as a linear matrix equation,
\begin{equation}
\mathbf{M} \mathbf{x} = \mathbf{b}.
\end{equation}
The constant vector $\mathbf{b}$ has 
elements $b_j = -A(i\varepsilon_j) \,(i \varepsilon_j)^{n+1}$
where $j = 0, 1, 2, ...., 2n + 1$. The column vector $\mathbf{x}$
represents the coefficients $\{p_i,q_i\}$.  In particular,
if we let $x_j = q_j$ for $j = 0, 1, ..., n$ and
$x_j = p_{j - n - 1}$ for $j = n+1, n+2, ..., 2n+1$, then
the constant matrix $\mathbf{M}$ has elements given by
$M_{j,k} = A(i\varepsilon_j) z^k$ for $k = 0, 1, ..., n$
and $M_{j,k} = -z^{k - n - 1}$ for $k = n+1, n+2, ..., 2n+1$.

Thus, given $2n+2$ positive frequencies, $\varepsilon_j$, and 
the values of $A(i\varepsilon_j)$ at those frequencies,
we can simply define the coefficient matrix
$\mathbf{M}$ and constant 
vector $\mathbf{b}$ that can be input into a linear equation
solver (such as from Lapack).  The solution vector
of length $2n+2$ contains the values of $\{ p_i, q_i \}$
that represent the constant coefficients in the 
Pade approximant of Eq.~(\ref{pade}).

