\chapter{Numerical first derivatives}

The following is a simple Taylor's series-based derivation
for numerically determined first derviatives.
We start with data on a finite grid of $\tau$-points,
\textit{i.e.} $\tau = 0^+, \Delta \tau, 2 \Delta \tau, \cdots$.
Assuming the $\tau \to 0^+$ first and second derivatives
are well-defined we have
\begin{eqnarray}
\label{dtau}
A(\Delta \tau) & = & A(0^+) + 
\Delta \tau \frac{\partial 
A(\tau)}{\partial \tau}{\bigg\vert}_{\tau \to 0^+} +
\frac{1}{2}(\Delta \tau)^2 
\frac{\partial^2 A(\tau)}{\partial \tau^2} {\bigg\vert}_{\tau \to 0^+} \\
\label{dtau2}
A(2\, \Delta \tau) & = &  A(0^+) + 
2 \Delta \tau 
\frac{\partial A(\tau)}{\partial \tau} {\bigg\vert}_{\tau \to 0^+} +
2 (\Delta \tau)^2 
\frac{\partial^2 A(\tau)}{\partial \tau^2} {\bigg\vert}_{\tau \to 0^+}
\end{eqnarray}
If we mutliply Eq.~(\ref{dtau}) by 4 and subtract Eq.~(\ref{dtau2}),
we get
\begin{equation}
4 A(\Delta \tau) - A(2 \Delta \tau) = 3 A(0^+) + 
2 \Delta \tau 
\frac{\partial A(\tau)}{\partial \tau} {\bigg\vert}_{\tau \to 0^+}
\end{equation}
Rearranging yields the final result
\begin{equation}
\frac{\partial A(\tau)}{\partial \tau} {\bigg\vert}_{\tau \to 0^+}
= \frac{2 A(\Delta\tau) - \frac{1}{2} A(2\,\Delta\tau)
- \frac{3}{2} A(0^+)}{\Delta \tau}.
\end{equation}

We are also interested in the $\tau \to 0^-$ limit.  In this
case the Taylor's series expansion yields
\begin{eqnarray}
\label{mdtau}
A(-\Delta \tau) & = & A(0^-) -
\Delta \tau \frac{\partial 
A(\tau)}{\partial \tau}{\bigg\vert}_{\tau \to 0^-} +
\frac{1}{2}(\Delta \tau)^2 
\frac{\partial^2 A(\tau)}{\partial \tau^2} {\bigg\vert}_{\tau \to 0^-} \\
\label{mdtau2}
A(-2\, \Delta \tau) & = &  A(0^-) -
2 \Delta \tau 
\frac{\partial A(\tau)}{\partial \tau} {\bigg\vert}_{\tau \to 0^-} +
2 (\Delta \tau)^2 
\frac{\partial^2 A(\tau)}{\partial \tau^2} {\bigg\vert}_{\tau \to 0^-}
\end{eqnarray}
As before, we mutliply Eq.~(\ref{mdtau}) by 4 and subtract 
Eq.~(\ref{mdtau2}),
to get
\begin{equation}
4 A(-\Delta \tau) - A(-2 \Delta \tau) = 3 A(0^-) - 
2 \Delta \tau 
\frac{\partial A(\tau)}{\partial \tau} {\bigg\vert}_{\tau \to 0^-}
\end{equation}
Rearranging yields the final result
\begin{equation}
\frac{\partial A(\tau)}{\partial \tau} {\bigg\vert}_{\tau \to 0^-}
= \frac{-2 A(-\Delta\tau) + \frac{1}{2} A(-2\,\Delta\tau)
+ \frac{3}{2} A(0^-)}{\Delta \tau}.
\end{equation}
