\chapter{$T$-matrices}

\section{General reprentation}

The particle-hole and particle-particle $T$-matrices
are $4 \times 4$ matrices in spin-space, \textit{i.e.}
the elements of the matrix $\mathbf{T}(\omega_m,\mathbf{r})$ are of the form
\begin{equation}
T_{\sigma_1 \sigma_2; \sigma_3 \sigma_4}(\omega_m,\mathbf{r})
\end{equation}
We write these matrices as the sum of analytic and
numerical pieces, $\mathbf{t}$ and $\tilde{\mathbf{T}}$, such that
\begin{equation}
\mathbf{T}(\omega_m,\mathbf{r}) \equiv \mathbf{t}(\omega_m,\mathbf{r})
+ \tilde{\mathbf{T}}(\omega_m,\mathbf{r}).
\end{equation}
The full $T$-matrix has an asymptotic expansion
\begin{equation}
\lim_{|\omega_m| \to \infty}\mathbf{T}(\omega_m,\mathbf{r}) 
=
 - \frac{\Delta \mathbf{T}(\mathbf{r})}{i \omega_m} 
+ \frac{\Delta \mathbf{T}^{\prime}(\mathbf{r})}
{(i \omega_m)^2} + O \left(\frac{1}{(i\omega_m)^3}\right).
\end{equation}
To ensure that the leading asymptotic dependence of $\mathbf{T}$
to order $1/(i \omega_m)^2$ is described by 
the analytic term, $\mathbf{t}$, we write
\begin{equation}
\begin{split}
\label{t-mat-form}
\mathbf{t}(\omega_m,\mathbf{r}) = & -(\Delta \mathbf{T}(\mathbf{r})
- \mathbf{d}_0(\mathbf{r}))\,R_0(\omega_m, y_{00}) -
 \mathbf{d}_0(\mathbf{r}) \,R_0(\omega_m, y_{01}) \\
& + (\Delta \mathbf{T}^{\prime}(\mathbf{r}) - \mathbf{d}_1(\mathbf{r}))
R_1(\omega_m, y_{10}) + \mathbf{d}_1(\mathbf{r}) R_1(\omega_m, y_{11})
\end{split}
\end{equation}
Here, $\mathbf{d}_0(\mathbf{r})$ and 
$\mathbf{d}_1(\mathbf{r})$ are matrices of the
same rank as $\mathbf{T}$.  Since they are frequency independent
and $R_0$ and $R_1$ have $1/(i\omega_m)$ and $1/(i \omega_m)^2$
asymptotic dependencies respectively, 
the above form guarantees that
$\mathbf{t}$ accounts for the leading asymptotic frequency
dependence of $\mathbf{T}$, irrespective of the
values for $\mathbf{d}_0(\mathbf{r})$ and 
$\mathbf{d}_1(\mathbf{r})$.
We utilize the freedom to choose $\mathbf{d}_0(\mathbf{r})$ and 
$\mathbf{d}_1(\mathbf{r})$
such that
\begin{eqnarray}
\label{t-mat-tau0}
\mathbf{t}(\tau \to 0^+, \mathbf{r}) & = & 
\mathbf{T}(\tau \to 0^+, \mathbf{r}) \\
\mathbf{t}^{\prime}(\tau \to 0^+, \mathbf{r}) & = & 
\mathbf{T}^{\prime}(\tau \to 0^+, \mathbf{r})
\end{eqnarray}
This choice is useful
for stabilizing the convolution of $T$-matrices and
Greens functions that occur in the evaluation of the
self-energy.  In particular, this choice guarantees that
the leading frequency dependence of the convolution is
described by the convolution of the analytic parts of
the Green's function and $T$-matrices.

Combining Eq.~(\ref{t-mat-form}) and Eq.~(\ref{t-mat-tau0}) yields
\begin{equation}
\begin{split}
\mathbf{T}(\tau \to 0^+, \mathbf{r}) = &
 -(\Delta \mathbf{T}(\mathbf{r})
- \mathbf{d}_0(\mathbf{r}))\,R_0(\tau \to 0^+, y_{00}) -
 \mathbf{d}_0(\mathbf{r}) \,R_0(\tau \to 0^+, y_{01}) \\
& + (\Delta \mathbf{T}^{\prime}(\mathbf{r}) - \mathbf{d}_1(\mathbf{r}))
R_1(\tau \to 0^+, y_{10}) + \mathbf{d}_1(\mathbf{r}) 
R_1(\tau \to 0^+, y_{11})
\end{split}
\end{equation}
Since we have
\begin{eqnarray}
R_0(\tau \to 0^+, y_{0j}) & = & \frac{-1}{2}\frac{\sinh(y_{0j} \beta/2)}
{\sinh(y_{0j} \beta/2)} \\
& = & \frac{-1}{2} \\
R_1(\tau \to 0^+, y_{1j}) & = & \frac{-1}{2 y_{1j}}
\frac{\cosh(y_{1j} \beta /2)}{\sinh(y_{1j} \beta /2)} \\
& = & \frac{-1}{2 y_{1j}}\frac{1}{\tanh(y_{1j} \beta /2)}
\end{eqnarray}
\begin{equation}
\begin{split}
\mathbf{T}(\tau \to 0^+, \mathbf{r}) = &
 -(\Delta \mathbf{T}(\mathbf{r})
- \mathbf{d}_0(\mathbf{r}))\,\left(-\,\frac{1}{2}\right) -
 \mathbf{d}_0(\mathbf{r}) \,\left(-\,\frac{1}{2}\right) \\
& + (\Delta \mathbf{T}^{\prime}(\mathbf{r}) - \mathbf{d}_1(\mathbf{r}))
\left(-\,\frac{1}{2 y_{10}} \frac{1}{\tanh(y_{10} \beta/2)}\right) \\
& + \mathbf{d}_1(\mathbf{r}) 
\left(-\,\frac{1}{2 y_{11}} \frac{1}{\tanh(y_{11} \beta/2)}\right)
\end{split}
\end{equation}
which is rewritten as
\begin{equation}
\begin{split}
\mathbf{T}(\tau \to 0^+, \mathbf{r})  &
- \frac{1}{2} \Delta \mathbf{T}(\mathbf{r}) + 
\frac{\Delta \mathbf{T}^{\prime}(\mathbf{r})}
{2y_{10} \tanh(y_{10} \beta /2)} \\
= & \left( \frac{1}{2y_{10} \tanh(y_{10} \beta /2)} -
\frac{1}{2y_{11} \tanh(y_{11} \beta /2)} \right) \,\mathbf{d}_1(\mathbf{r}).
\end{split}
\end{equation}
Finally, we solve for $\mathbf{d}_1(\mathbf{r})$ to get
\begin{equation}
\mathbf{d}_1(\mathbf{r}) = 
\frac{
\mathbf{T}(\tau \to 0^+, \mathbf{r}) 
- \frac{1}{2} \Delta \mathbf{T}(\mathbf{r}) + 
{\displaystyle \frac{\Delta \mathbf{T}^{\prime}(\mathbf{r})}
{2y_{10} \tanh(y_{10} \beta /2)}}
}
{\displaystyle \left(
 \frac{1}{2y_{10} \tanh(y_{10} \beta /2)} -
\frac{1}{2y_{11} \tanh(y_{11} \beta /2)} \right)
}
\end{equation}

We also have
\begin{equation}
\begin{split}
\mathbf{T}^{\prime}(\tau \to 0^+, \mathbf{r}) = &
 -(\Delta \mathbf{T}(\mathbf{r})
- \mathbf{d}_0(\mathbf{r}))\,R_0^{\prime}(\tau \to 0^+, y_{00}) -
 \mathbf{d}_0(\mathbf{r}) \,R_0^{\prime}(\tau \to 0^+, y_{01}) \\
& + (\Delta \mathbf{T}^{\prime}(\mathbf{r}) - \mathbf{d}_1(\mathbf{r}))
R_1^{\prime}(\tau \to 0^+, y_{10}) + \mathbf{d}_1(\mathbf{r}) 
R_1^{\prime}(\tau \to 0^+, y_{11})
\end{split}
\end{equation}
The derivatives of the $R$-functions are as follows
\begin{eqnarray}
\frac{\partial R_0(\tau)}{\partial \tau}\bigg\vert_{\tau \to 0^+} & = &
-\frac{1}{2}\frac{1}{\sinh(y_{0j} \beta/2)} \;
\frac{\partial}{\partial \tau}
\sinh(y_{0j}(\beta/2 - \tau))\, \bigg\vert_{\tau \to 0^+} \\
& = & -\frac{1}{2}\frac{1}{\sinh(y_{0j} \beta/2)} )(-y_{0j})
\cosh(y_{0j}(\beta/2 - \tau))\, \bigg\vert_{\tau \to 0^+} \\
& = & \frac{y_{0j}}{2 \tanh(y_{0j} \beta /2)} \\
\frac{\partial R_1(\tau)}{\partial \tau}\bigg\vert_{\tau \to 0^+} & = &
-\frac{1}{2 y_{1j}}\frac{1}{\sinh(y_{1j} \beta/2)} \;
\frac{\partial}{\partial \tau}
\cosh(y_{1j}(\beta/2 - \tau))\, \bigg\vert_{\tau \to 0^+} \\
& = & -\frac{1}{2 y_{1j}}\frac{1}{\sinh(y_{1j} \beta/2)} )(-y_{1j})
\sinh(y_{1j}(\beta/2 - \tau))\, \bigg\vert_{\tau \to 0^+} \\
& = & \frac{1}{2}
\end{eqnarray}
Therefore
\begin{equation}
\begin{split}
\mathbf{T}^{\prime}(\tau \to 0^+, \mathbf{r}) = &
 -(\Delta \mathbf{T}(\mathbf{r})
- \mathbf{d}_0(\mathbf{r}))\,\frac{y_{00}}{2 \tanh(y_{00} \beta/2)} -
 \mathbf{d}_0(\mathbf{r}) \,\frac{y_{01}}{2 \tanh(y_{01} \beta/2)} \\
& + (\Delta \mathbf{T}^{\prime}(\mathbf{r}) - \mathbf{d}_1(\mathbf{r}))
\left( \frac{1}{2} \right) + \mathbf{d}_1(\mathbf{r}) 
\left( \frac{1}{2} \right)
\end{split}
\end{equation}
which is rewritten as
\begin{equation}
\begin{split}
\mathbf{T}^{\prime}(\tau \to 0^+, \mathbf{r}) & + 
\frac{
y_{00}\;\Delta \mathbf{T}(\mathbf{r})}
{2 \tanh(y_{00} \beta /2)}
-
\frac{1}{2}\,\Delta \mathbf{T}^{\prime}(\mathbf{r}) \\
 & = \left( \frac{y_{00}}{2\tanh(y_{00}\beta /2)}
 - \frac{y_{01}}{2 \tanh(y_{01}\beta /2)} \right) \mathbf{d}_0(\mathbf{r}).
\end{split}
\end{equation}
Finally we solve for $\mathbf{d}_0(\mathbf{r})$ to get
\begin{equation}
 \mathbf{d}_0(\mathbf{r}) =
\frac{
\mathbf{T}^{\prime}(\tau \to 0^+, \mathbf{r}) + 
{\displaystyle
\frac{
y_{00}\;\Delta \mathbf{T}(\mathbf{r})}
{2 \tanh(y_{00} \beta /2)}}
-
\frac{1}{2}\,\Delta \mathbf{T}^{\prime}(\mathbf{r})}
{\displaystyle
\left(\frac{y_{00}}{2\tanh(y_{00}\beta /2)}
- \frac{y_{01}}{2 \tanh(y_{01}\beta /2)} \right)
}
\end{equation}

The notation we have used thus far is cumbersome for
programming.  To simplify we adopt the following notation
for $\mathbf{t}$:
\begin{equation}
\mathbf{t}(\omega_m,\mathbf{r}) = \sum_{ij} \mathbf{d}_{ij}(\mathbf{r})\,
R_{ij}(\omega_m)
\end{equation}
where
\begin{equation}
R_{ij}(\omega_m) \equiv R_i(\omega_m,y_{ij})
\end{equation}
The coefficient matrices $\mathbf{d}_{ij}(\mathbf{r})$
are given by
\begin{eqnarray}
\mathbf{d}_{01}(\mathbf{r}) & = &
\frac{-\left(
\mathbf{T}^{\prime}(\tau \to 0^+, \mathbf{r}) + 
{\displaystyle
\frac{
y_{00}\;\Delta \mathbf{T}(\mathbf{r})}
{2 \tanh(y_{00} \beta /2)}}
-
\frac{1}{2}\,\Delta \mathbf{T}^{\prime}(\mathbf{r})\right)}
{\displaystyle
\left(\frac{y_{00}}{2\tanh(y_{00}\beta /2)}
- \frac{y_{01}}{2 \tanh(y_{01}\beta /2)} \right)
} \\
\mathbf{d}_{00}(\mathbf{r}) & = & -\Delta \mathbf{T}(\mathbf{r})
 - \mathbf{d}_{01}(\mathbf{r}) \\
\mathbf{d}_{11}(\mathbf{r}) & = &
\frac{
\mathbf{T}(\tau \to 0^+, \mathbf{r}) 
- \frac{1}{2} \Delta \mathbf{T}(\mathbf{r}) + 
{\displaystyle \frac{\Delta \mathbf{T}^{\prime}(\mathbf{r})}
{2y_{10} \tanh(y_{10} \beta /2)}}
}
{\displaystyle \left(
 \frac{1}{2y_{10} \tanh(y_{10} \beta /2)} -
\frac{1}{2y_{11} \tanh(y_{11} \beta /2)} \right)
} \\
\mathbf{d}_{10}(\mathbf{r}) & = & \Delta \mathbf{T}^{\prime}(\mathbf{r})
 - \mathbf{d}_{11}(\mathbf{r}) 
\end{eqnarray}