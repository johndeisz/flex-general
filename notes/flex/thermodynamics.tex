\chapter{Thermodynamics}
\label{chapter:thermodynamics}

\section{$\Omega$: non-$\Phi$ terms}
We evaluate the Gibbs free energy using the following formula,
\begin{equation}
\label{omega}
\Omega(T,\mu) = \Omega_c -\frac{1}{2} \mathrm{Tr}\,\left[\Sigma G +
 \ln(-[G^{(0)}]^{-1}  + \Sigma)\right]     +
\Phi[G]
\end{equation}
Here $\Omega_c$ is just a constant term that arrises 
from generalizing diagonal elements of
the single-particle Hamiltonian to include hole operators
in the basis; its value is given by
\begin{eqnarray}
\Omega_c & = & - \frac{1}{2} \sum_{i,\sigma} \mu \\
         & = & - N_{sites} \mu.
\end{eqnarray}
The factor of $\frac{1}{2}$ in Eq.~(\ref{omega}) is used
to compensate for the use of an overcomplete basis set that
includes both particle and hole operators.

Next we consider the term
\begin{equation}
-\frac{1}{2} \mathrm{Tr} \,\Sigma G = 
-\frac{1}{2}\mathrm{Tr}\, \Sigma^{(1)}G 
-\frac{1}{2} \mathrm{Tr} \,\Sigma^{ph}G.
\end{equation}
The first term, $-\frac{1}{2}\mathrm{Tr}\, \Sigma^{(1)}G$ is equal
to $-\frac{1}{2}$ times the Hartree energy and is evaluated as
part of the interaction energy.  An expression for this
appears in Chapter~\ref{first-order}.
The second term, $-\frac{1}{2}\mathrm{Tr}\, \Sigma^{ph}G$ is
equal to $-\frac{1}{2}$ times the correlation energy.
It is also evaluated as part of the interaction energy.
Notes on its evaluation appears in Chapter~\ref{chapter:particle-hole}.

Next we consider the term
\begin{equation}
\label{log_terms}
 -\frac{1}{2}\mathrm{Tr}\,
 \ln(-[G^{(0)}]^{-1}  + \Sigma).
\end{equation}
Here we include the first-order self-energy, $\Sigma^{(1)}$,
in $[G^{(0)}]^{-1}$ so that $\Sigma$ only includes
$\Sigma^{ph}$.  We rewrite Eq.~(\ref{log_terms}) as follows:
\begin{eqnarray}
& = & -\frac{1}{2}\mathrm{Tr} \, \ln(-[G^{(0)}]^{-1})
  - \frac{1}{2}\mathrm{Tr}\,\ln(1 - G^{(0)} \Sigma) \\
\label{new_log}
& = & -\frac{1}{2}\mathrm{Tr} \, \ln(-[G^{(0)}]^{-1})
  - \frac{1}{2}\mathrm{Tr}\,\left[\ln(1 - G^{(0)} \Sigma) + G \Sigma\right] 
   + \frac{1}{2}\mathrm{Tr} \, G \Sigma 
\end{eqnarray}
The three traces in Eq.~(\ref{new_log}) are performed separately.

The first trace is simply evaluated as
\begin{equation}
-\frac{1}{2} \mathrm{Tr}\, \ln [G^{(0)}]^{-1} = - \frac{T}{2} 
\sum_{\mathbf{k},i}
\ln(1  + e^{-E_i(\mathbf{k})\beta})
\end{equation}
where $E_i(\mathbf{k})$ are the four eigenvalues of
the matrix $H^{(0)}(\mathbf{k}) + \Sigma^{(1)}$.

The second trace is evaluated numerically in frequency space.
No special treatment of the cut-off is performed as
the combined asymptotic dependence
of the terms in the second trace
goes as $1 / (i \varepsilon_n)^4$.
To evaluate the log term we diagonalize $G^{(0)}\Sigma$ for
each $\varepsilon, \mathbf{k}$.  Thus, if
\begin{equation}
\Lambda_i(\varepsilon_n, \mathbf{k}) = 
\mathrm{eigvals}(G^{(0)}(\varepsilon_n,\mathbf{k})
\Sigma(\varepsilon_n,\mathbf{k}))\;\;\;i = 0,1,2,3
\end{equation}
then this trace is given by
\begin{equation}
-\frac{T}{2} \sum_{\varepsilon_n,\mathbf{k},i}
\left( \ln(1 - \Lambda_i(\varepsilon_n,\mathbf{k})) +
\left[G(\varepsilon_n,\mathbf{k})\Sigma(\varepsilon_n,\mathbf{k})
\right]_{i,i} \right)
\end{equation}

The third trace is just $\frac{1}{2}\mathrm{Tr}\,\Sigma G$.  We have
already described the evaluation of this trace in this
section.

\section{$\Omega$: $\Phi$ terms}

\subsection{$\Phi^{(1)}$}

As stated in Sec.~(\ref{first-order-self-energy}), 
\begin{equation}
\Phi^{(1)} = \frac{T}{4} \sum_{\mathbf{r}_1} \int_0^{\beta}d\tau_1
\sum_{\alpha_0 \alpha_1 \beta_0 \beta_1}
\Gamma^{(0)}_{\alpha_0 \alpha_1; \beta_0 \beta_1}
G_{\beta_1 \alpha_1}(\tau_1,\mathbf{r}_1;\tau,\mathbf{r}_1)
\,G_{\beta_0 \alpha_0}(\tau_1,\mathbf{r}_1;\tau,\mathbf{r}_1).
\end{equation}
For a translationally invariant system this becomes
\begin{equation}
\Phi^{(1)} = \frac{1}{4}\sum_{\alpha_0 \alpha_1\beta_0 \beta_1}
\Gamma^{(0)}_{\alpha_0 \alpha_1; \beta_0 \beta_1}
G_{\beta_1 \alpha_1}(\tau=0,\mathbf{r}=0)\,
G_{\beta_0 \alpha_0}(\tau=0,\mathbf{r}=0).
\end{equation} 
Summing over the spin indices gives
\begin{eqnarray}
\Phi^{(1)} & = & \frac{1}{4}\, U ( 
G_{11}G_{00} - G_{10}G_{01} - G_{00}G_{33} 
+ G_{30}G_{03} + G_{10}G_{32} - G_{30}G_{12} \\
\nonumber
& & + G_{11}G_{00} - G_{01}G_{10} - G_{11}G_{22} 
+ G_{21}G_{12} + G_{01}G_{23} - G_{03}G_{21} \\
\nonumber
& & + G_{22}G_{33} - G_{32}G_{23} - G_{22}G_{11}
+ G_{21}G_{12} + G_{32}G_{10} - G_{12}G_{30} \\
\nonumber
& &+ G_{33}G_{22} - G_{23}G_{23} - G_{33}G_{00}
+ G_{30}G_{03} + G_{23}G_{01} - G_{03}G_{21} 
) \\
& = & 2\, U \left(G_{00}(0^-)G_{11}(0^-) -
 G_{10}(0)G_{01}(0) + G_{30}(0)G_{03}(0) \right)
\end{eqnarray}

\subsection{$\Phi^{(2)}$}

We treat the second order diagram as if it is part of the
particle-hole self-energy.  The corresponding $\Phi$ diagram
is given by
\begin{equation}
\Phi^{(2)} = - \frac{T}{2} \sum_{\alpha_0 \beta_0; \alpha_1 \beta_1}
 \int dx_1\,dx_2 \; \chi^{ph}_{\alpha_1 \beta_1; \alpha_0 \beta_0}(x_1,x_0)
\, \chi^{ph}_{\alpha_0 \beta_0; \alpha_1 \beta_1}(x_0,x_1).
\end{equation}
Taking translational invariance into account gives
\begin{equation}
\Phi^{(2)} = - \frac{1}{2} \sum_{\alpha_0 \beta_0; \alpha_1 \beta_1}
\int^{\beta}_0 d\tau \sum_{\mathbf{r}} 
\chi^{ph}_{\alpha_1 \beta_1; \alpha_0 \beta_0}(\tau, \mathbf{r})\,
\chi^{ph}_{\alpha_0 \beta_0; \alpha_1 \beta_1}(-\tau, -\mathbf{r}).
\end{equation}
We use the definition of $\chi^{ph}$,
\begin{equation}
\chi^{ph}_{\alpha_1 \beta_1; \alpha_0 \beta_0}(x_1,x_0) 
 =  (-1) \sum_{\gamma_1 \delta_1}
\Gamma^{(0)ph}_{\alpha_1\beta_1; \delta_1\gamma_1}\,
G_{\gamma_1 \alpha_0}(x_1,x_0) \; G_{\beta_0 \delta_1}(x_0,x_1),
\end{equation}
to rewrite $\Phi^{(2)}$ as
\begin{equation}
\begin{split}
\Phi^{(2)} =  - \frac{1}{2} \sum_{\alpha_0 \beta_0 \gamma_0 \delta_0;
\alpha_1 \beta_1 \gamma_1 \delta_1} \int^{\beta}_0 \sum_{\mathbf{r}}\; & 
\Gamma^{(0)ph}_{\alpha_1 \beta_1; \delta_1 \gamma_1}
G_{\gamma_1 \alpha_0}(\tau, \mathbf{r}) \,
G_{\beta_0 \delta_1}(-\tau, - \mathbf{r}) \\
& \times
\Gamma^{(0)ph}_{\alpha_0 \beta_0; \delta_0 \gamma_0}
G_{\gamma_0 \alpha_1}(-\tau, -\mathbf{r}) \,
G_{\beta_1 \delta_0}(\tau, \mathbf{r}).
\end{split}
\end{equation}
The $\tau$-integration is performed using Simpson's rule.

\subsection{$\Phi_{ph}$}

The last term to evaluate is
\begin{eqnarray}
\Phi_{ph} = -T \sum_{n > 2} \frac{1}{n}
\; \mathrm{Tr} \; (\chi^{ph})^n.
\end{eqnarray}
Using the identity
\begin{equation}
\ln (1 - x) = -(x + \frac{x^2}{2} + \frac{x^3}{3} \cdots)
\end{equation}
the above becomes
\begin{equation}
\Phi_{ph} = T \left(\mathrm{Tr} \; \ln(1 - \chi^{ph}) + \chi^{ph} + 
(\chi^{ph})^2 /2\right).
\end{equation}  
$\chi$ is a $16 \times 16$ matrix, 
so if we diagonalize $\chi$
such that
\begin{equation}
\Lambda_i(\varepsilon_n, \mathbf{k}) = 
\mathrm{eigvals}\,( \chi(\omega_m, \mathbf{k})),\;\;\;i = 1,2,3,4
\end{equation}
then the expression becomes
\begin{equation}
\Phi_{ph} = 
 T \sum_{\omega_m,\mathbf{k},i} \ln(1 - \Lambda_i(\omega_m,\mathbf{k}))
 +  \Lambda_i(\omega_m,\mathbf{k}) + 
\frac{\Lambda_i^2(\omega_m,\mathbf{k})}{2}.
\end{equation}
