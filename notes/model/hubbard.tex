\chapter{Hubbard Hamiltonian}
\label{chapter:hubbard}

\section{Normal terms}
\label{section:hubbard-normal}
The Hamiltonian for the Hubbard model is defined as 
\begin{equation}
\begin{split}
H - \mu N &= -\sum_{<\mathbf{r},\mathbf{r}^{\prime}>,\nu\nu^{\prime}\sigma}  
\left(t_{\nu\nu^{\prime}}(\mathbf{r},\mathbf{r}^{\prime})
e^{i \vec{\chi}\cdot (\mathbf{r}-\mathbf{r}^{\prime})}  
c^{\dagger}_{\vec{r}\nu\sigma}c_{\vec{r}^{\prime}\nu^{\prime}\sigma} + 
\mathrm{h.c.} \right) \\
& - \sum_{\mathbf{r}\nu\sigma} \mu \; c^{\dagger}_{\mathbf{r}\nu\sigma}c_{\vec{r}\nu\sigma} 
 -\sum_{\mathbf{r}\nu\sigma\sigma^{\prime}} \mathbf{h}\cdot 
\vec{\sigma}_{\sigma\sigma^{\prime}} 
\, c^{\dagger}_{\vec{r}\nu\sigma} c_{\vec{r}\nu\sigma^{\prime}} \\
& + \sum_{\mathbf{r}\nu\nu^{\prime}\sigma\sigma^{\prime}} U_{\nu\nu^{\prime}} 
c^{\dagger}_{\mathbf{r}\nu\sigma}c_{\mathbf{r}\nu\sigma}
c^{\dagger}_{\mathbf{r}\nu^{\prime}\sigma^{\prime}}c_{\mathbf{r}\nu^{\prime}\sigma^{\prime}} 
+ \sum_{\mathbf{r}\nu\nu^{\prime}\sigma\sigma^{\prime}} J_{\nu\nu^{\prime}}
c^{\dagger}_{\mathbf{r}\nu\sigma}c_{\mathbf{r}\nu^{\prime}\sigma^{\prime}}
c^{\dagger}_{\mathbf{r}\nu^{\prime}\sigma^{\prime}}c_{\mathbf{r}\nu\sigma}
\end{split}
\end{equation}
Along with the hopping amplitude, 
$t_{\nu\nu^{\prime}}(\mathbf{r},\mathbf{r}^{\prime})$, 
we include a phase factor with a term
$\vec{\chi}\cdot(\mathbf{r}-\mathbf{r}^{\prime}) 
\equiv 2 \pi (\phi_{B,x} / \phi_o)( (x-x^{\prime})/ L_x)$
+ similar terms for the $y$ and $z$ directions.
This phase factor characterizes the Aharanov-Bohm
phase acqured when an electron makes one complete revolution
around the lattice in the $x$, $y$ or
$z$ directions and, in doing
so, completes a circuit that encloses a magnetic
flux, $\phi_{B,x}$, $\phi_{B,y}$ and/or $\phi_{B,z}$. 

The chemical potential, $\mu$, regulates the number of
electrons in the system.  A uniform magnetic field, $\mathbf{h}$,
polarizes the electron spin.
Direct coulomb, $U$, and exchange terms, $J$, characterize the 
effective interaction between electrons on the same atomic site.

The transformation from ${\mathbf k}$-space to ${\mathbf r}$-space
is made by subsituting 
\begin{eqnarray}
\label{c-forward}
c^{\dagger}_{{\mathbf k}\nu\sigma} & = & \frac{1}{\sqrt{N}}
\sum_{{\mathbf r}} \exp(i \mathbf{k}\cdot\mathbf{r}) \; 
c^{\dagger}_{{\mathbf r}\nu\sigma} \\
\label{c-backwards}
c^{\dagger}_{{\mathbf r}\nu\sigma} & = & \frac{1}{\sqrt{N}}
\sum_{{\mathbf k}} \exp(-i \mathbf{k}\cdot\mathbf{r}) \; 
c^{\dagger}_{{\mathbf k}\nu\sigma} 
\end{eqnarray}
with equivalent relations for the annihilation operators
obtained by taking the 
Hermetian conjugates of Eqs~(\ref{c-forward}) and (\ref{c-backwards}).
Local terms transform as follows:
\begin{eqnarray}
\sum_{\mathbf{r}} c^{\dagger}_{\mathbf{r}\nu\sigma} c_{\mathbf{r}\nu\sigma} & = &
\frac{1}{N} \sum_{\mathbf{r}} \sum_{\mathbf{k},\mathbf{k}^{\prime}}
e^{-i \mathbf{k}\cdot\mathbf{r}} e^{i \mathbf{k}^{\prime}\cdot\mathbf{r}}
c^{\dagger}_{\mathbf{k}\nu\sigma} c_{\mathbf{k}^{\prime}\nu\sigma} \\
& = & \sum_{\mathbf{k},\mathbf{k}^{\prime}} 
\delta_{\mathbf{k},\mathbf{k}^{\prime}}
c^{\dagger}_{\mathbf{k}\nu\sigma} c_{\mathbf{k}^{\prime}\nu\sigma} \\
& = & \sum_{\mathbf{k}} c^{\dagger}_{\mathbf{k}\nu\sigma} c_{\mathbf{k}\nu\sigma}
\end{eqnarray}  
Likewise we obtain a simple form for the kinetic energy terms
when they are expressed in momentum space:
\begin{eqnarray}
\mathrm{KE} & = &
-\sum_{<\mathbf{r}\mathbf{r}^{\prime}>\nu\nu^{\prime}\sigma} 
t_{\nu\nu^{\prime}}(\mathbf{r},\mathbf{r}^{\prime})
e^{i\vec{\chi}\cdot(\mathbf{r}-\mathbf{r}^{\prime})}
c^{\dagger}_{\mathbf{r}\nu\sigma}c_{\mathbf{r}^{\prime}\nu^{\prime}\sigma} + 
 t_{\nu\nu^{\prime}}^*(\mathbf{r},\mathbf{r}^{\prime}) 
e^{-i\vec{\chi}\cdot (\mathbf{r}-\mathbf{r}^{\prime})}
c^{\dagger}_{\mathbf{r}^{\prime}\nu^{\prime}\sigma}c_{\mathbf{r}\nu\sigma} \\
& = & -\sum_{<\mathbf{r}\mathbf{r}^{\prime}>\nu\nu^{\prime}\sigma} 
t_{\nu\nu^{\prime}}(\mathbf{r},\mathbf{r}^{\prime})
e^{i\vec{\chi}\cdot(\mathbf{r}-\mathbf{r}^{\prime})}
c^{\dagger}_{\mathbf{r}\nu\sigma}c_{\mathbf{r}^{\prime}\nu^{\prime}\sigma} +
 t_{\nu^{\prime}\nu}^*(\mathbf{r}^{\prime},\mathbf{r})
e^{-i\vec{\chi}\cdot(\mathbf{r}^{\prime}-\mathbf{r})}
c^{\dagger}_{\mathbf{r}\nu\sigma}c_{\mathbf{r}^{\prime}\nu^{\prime}\sigma} \\
& = & -\frac{1}{2} \sum_{\mathbf{r}^{\prime}\vec{\delta}\nu\nu^{\prime}\sigma} 
t_{\nu\nu^{\prime}}(\vec{\delta}) e^{i\vec{\chi}\cdot \vec{\delta}} 
c^{\dagger}_{\mathbf{r}^{\prime}+\vec{\delta}\nu\sigma} c_{\vec{r}^{\prime}\nu^{\prime}\sigma}
+ t_{\nu^{\prime}\nu}^*(-\vec{\delta}) e^{ i \vec{\chi} \cdot \vec{\delta}} 
c^{\dagger}_{\mathbf{r}^{\prime}+ \vec{\delta}\nu\sigma} c_{\mathbf{r}^{\prime}\nu^{\prime}\sigma} \\
& = & - \sum_{\mathbf{r}^{\prime}\vec{\delta}} t_{\nu\nu^{\prime}}(\vec{\delta}) 
e^{i \vec{\chi}\cdot \vec{\delta}} 
c^{\dagger}_{\mathbf{r}^{\prime}+\vec{\delta}\nu\sigma} c_{\vec{r}^{\prime}\nu^{\prime}\sigma}
\end{eqnarray}
since $t_{\nu^{\prime}\nu}^*(-\vec{\delta}) = t_{\nu\nu^{\prime}}(\vec{\delta})$.
Substituation of Eq.~(\ref{c-backwards}) gives
\begin{eqnarray}
& = &-\frac{1}{N}
\sum_{\mathbf{r}\vec{\delta}\nu\nu^{\prime}\sigma \mathbf{k}\mathbf{k}^{\prime}}
t_{\nu\nu^{\prime}}(\vec{\delta})e^{i\vec{\chi}\cdot \vec{\delta}} 
e^{-i \mathbf{k}\cdot (\mathbf{r} + \vec{\delta})}
e^{i \mathbf{k}^{\prime} \cdot \mathbf{r}}
     c^{\dagger}_{\mathbf{k}\nu\sigma}
  c_{\mathbf{k}^{\prime}\nu^{\prime}\sigma} \\
& = & - \sum_{\vec{\delta}\nu\nu^{\prime}\mathbf{k},\mathbf{k}^{\prime}}
t_{\nu\nu^{\prime}}(\vec{\delta})e^{i \vec{\chi}\cdot \vec{\delta}} 
 e^{-i \mathbf{k}\cdot \vec{\delta}} \delta_{\mathbf{k},\mathbf{k}^{\prime}}
     c^{\dagger}_{\mathbf{k}\nu\sigma}
  c_{\mathbf{k}^{\prime}\nu^{\prime}\sigma} \\
& = & - \sum_{\mathbf{k}\vec{\delta}\nu\nu^{\prime}}  
e^{-i \mathbf{k}\cdot \vec{\delta}} 
t_{\nu\nu^{\prime}}(\vec{\delta})e^{i \vec{\chi}\cdot\vec{\delta}} 
  c^{\dagger}_{\mathbf{k}\nu\sigma}
  c_{\mathbf{k}\nu^{\prime}\sigma} \\
& = & \sum_{\mathbf{k}\nu\nu^{\prime}} \epsilon_{\nu\nu^{\prime}}(\mathbf{k})
  c^{\dagger}_{\mathbf{k}\nu\sigma}
  c_{\mathbf{k}\nu^{\prime}\sigma}
 \end{eqnarray}
where
\begin{equation}
\epsilon_{\nu\nu^{\prime}}(\mathbf{k}) = - \sum_{\vec{\delta}} 
e^{-i \mathbf{k}\cdot\vec{\delta}} e^{i \vec{\chi}\cdot\vec{\delta}}  
t_{\nu\nu^{\prime}}(\vec{\delta})
\end{equation}
is the Fourier transform of the hopping integrals.
Thus, if we write all the one-electron terms in $\mathbf{k}$-space
the Hamiltonian becomes
\begin{equation}
H_0 - \mu N =  \sum_{\mathbf{k}\nu\nu^{\prime}\sigma\sigma^{\prime}} 
((\epsilon_{\nu\nu^{\prime}}(\mathbf{k}) - \mu\delta_{\nu\nu^{\prime}})
\delta_{\sigma\sigma^{\prime}} -
 \mathbf{h}\cdot\vec{\sigma}_{\sigma\sigma^{\prime}}\delta_{\nu\nu^{\prime}}) 
c^{\dagger}_{\mathbf{k}\nu\sigma}
 c_{\mathbf{k}\nu^{\prime}\sigma^{\prime}} 
\end{equation} 

\section{External pair-field coupling terms}
\label{section:pair-field}

Let
\begin{eqnarray}
H_{p} & = & - \frac{\sqrt{N_l}}{2} 
\sum_{\mathbf{r}\mathbf{r}^{\prime}\nu\nu^{\prime}\sigma\sigma^{\prime}} 
h_p \left( \Psi_{\nu\sigma\nu^{\prime}\sigma^{\prime}}(\mathbf{r},\mathbf{r}^{\prime})
c_{\mathbf{r}\nu\sigma}
c_{\mathbf{r}^{\prime}\nu^{\prime}\sigma^{\prime}} + \mathrm{h.c.}\right) \\
 &  = & - \frac{\sqrt{N_l}}{2} \sum_{\mathbf{r}\mathbf{r}^{\prime}\nu\nu^{\prime}\sigma\sigma^{\prime}} 
h_p \left(\Psi_{\nu\sigma\nu^{\prime}\sigma^{\prime}}(\mathbf{r},\mathbf{r}^{\prime})  
c_{\mathbf{r}\nu\sigma}
c_{\mathbf{r}^{\prime}\nu^{\prime}\sigma^{\prime}} + 
\Psi_{\nu\sigma\nu^{\prime}\sigma^{\prime}}^*(\mathbf{r},\mathbf{r}^{\prime}) 
c^{\dagger}_{\mathbf{r}^{\prime}\nu^{\prime}\sigma^{\prime}}
 c^{\dagger}_{\mathbf{r}\nu\sigma}\right) \\
& = &  - \frac{\sqrt{N_l}}{2} 
\sum_{\mathbf{r}\mathbf{r}^{\prime}\nu\nu^{\prime}\sigma\sigma^{\prime}} 
h_p \left(\Psi_{\nu\sigma\nu^{\prime}\sigma^{\prime}}(\mathbf{r},\mathbf{r}^{\prime})  
c_{\mathbf{r}\nu\sigma}
c_{\mathbf{r}^{\prime}\nu^{\prime}\sigma^{\prime}} + 
\Psi_{\nu^{\prime}\sigma^{\prime}\nu\sigma}^*(\mathbf{r}^{\prime},\mathbf{r}) 
 c^{\dagger}_{\mathbf{r}\nu\sigma}c^{\dagger}_{\mathbf{r}^{\prime}\nu^{\prime}\sigma^{\prime}}
  \right)
\end{eqnarray}
Here $\Psi$ is the normalized pairing wave function
where the normalization is given by
\begin{equation}
S(\nu_{max})  =  1 
\end{equation}
where $\nu_{max}$ corresponds to the band index 
that maximizes the function
\begin{equation}
S(\nu) = \sum_{\mathbf{r}\mathbf{r}^{\prime}\nu^{\prime}\sigma\sigma^{\prime}} | 
\Psi_{\nu\sigma\nu^{\prime}\sigma^{\prime}}(\mathbf{r},\mathbf{r}^{\prime}) |^2 
\end{equation}
This normalization choice is made so that (1) for a given field, $h_p$, 
identical results are obtained whether or not equivalent orbitals are
treated as distinct or not and (2) if non-active bands are introduced no
change is made to the normalization of the active bands.
$N_l$ is the number of lattice points
and the real constant $h_p$ describes the strength of the
pairing field.

Since 
$\{c_{\mathbf{r}\nu\sigma},c_{\mathbf{r}^{\prime}\nu^{\prime}\sigma^{\prime}}\}=0$ by the Fermi
commutation relations, only the antisymmetric part of
$\Psi$ effectively contributes to this term.
Forcing $\Psi$ to be antisymmetric gives us
$\Psi_{\nu^{\prime}\sigma^{\prime}\nu\sigma}(\mathbf{r}^{\prime},\mathbf{r}) =
- \Psi_{\nu\sigma\nu^{\prime}\sigma^{\prime}}(\mathbf{r},\mathbf{r}^{\prime})$.
Other than this,
the functional dependence of $\Psi$ may be arbitary, but we
will be interested in forms for $\Psi$ that represent
the dominant pairing symmetry for this model.  When there is
translational invariance we expect that
\begin{equation}
 | \Psi_{\nu\sigma\nu^{\prime}\sigma^{\prime}}(\mathbf{r}+\vec{\delta},
\mathbf{r}^{\prime}+\vec{\delta}) | =
|\Psi_{\nu\sigma\nu^{\prime}\sigma^{\prime}}(\mathbf{r},\mathbf{r}^{\prime})|
\end{equation}
so that
\begin{equation}
\Psi_{\nu\sigma\nu^{\prime}\sigma^{\prime}}(\mathbf{r}+\vec{\delta},
\mathbf{r}^{\prime}+\vec{\delta}) = 
e^{i\theta(\vec{\delta})} 
\Psi_{\nu\sigma\nu^{\prime}\sigma^{\prime}}(\mathbf{r},\mathbf{r}^{\prime}).
\end{equation}
(\textbf{Note: does $\theta$ need to have band and spin indices?'})
Since the pairing field must be single valued we require 
$\Psi_{\nu\sigma\nu^{\prime}\sigma^{\prime}}(\vec{r} + L_x\hat{x}, \vec{r}^{\prime} + L_x \hat{x}) = 
\Psi_{\nu\sigma\nu^{\prime}\sigma^{\prime}}(\vec{r}, \vec{r}^{\prime})$
with similar relations for center-of-mass displacements in the 
$y$ and $z$ directions.
We assume also that the spatial variation of the phase will
be as smooth as possible.  This leads to
\begin{equation}
\theta(\vec{\delta}) = 2 \pi (n_{p,x} \delta_x / L_x + n_{p,y} 
\delta_y / L_y
+ n_{p,z} \delta_z / L_z).
\end{equation}
$n_{p,x}$ is an integer that describes the number times
the phase rotates for each $\delta_x = L_x$ of displacement
and has only $L_x$ unique values which we take to be
given by $n_{p,x} = -L_x/2 + 1, -L_x/2 + 2, \cdots -1, 0, 1, \cdots L_x/2$.

Based on these considerations we write
\begin{equation}
\Psi_{\nu\sigma\nu^{\prime}\sigma^{\prime}}(\mathbf{r},\mathbf{r}^{\prime}) = 
\psi\left(\mathbf{r}_{cm}\right) 
\phi_{\nu\sigma\nu^{\prime}\sigma^{\prime}}(\mathbf{r} - \mathbf{r}^{\prime}).
\end{equation}
where the center of mass part, $\psi(\mathbf{r}_{cm})$, is given
by
\begin{equation}
\psi(\mathbf{r}_{cm}) = 
\frac{1}{\sqrt{N_l}}e^{i 2 \pi ( n_{p,x} x_{cm} / L_x + n_{p,y} y_{cm}/L_y + 
n_{p,z} z_{cm}/L_z)}
\end{equation}
and the normalization of $\phi$ is given by
\begin{eqnarray}
S(\nu_{max}) = \sum_{\mathbf{r}\mathbf{r}^{\prime}\nu^{\prime}\sigma\sigma^{\prime}} 
|\Psi_{\nu_{max}\sigma\nu^{\prime}\sigma^{\prime}}(\mathbf{r},\mathbf{r}^{\prime})|^2 
& = & 1 \\
\frac{1}{N_l}
\sum_{\mathbf{r}\mathbf{r}^{\prime}\nu^{\prime}\sigma\sigma^{\prime}} 
|\phi_{\nu_{max}\sigma\nu_{\prime}\sigma^{\prime}}(\mathbf{r} - \mathbf{r}^{\prime}) |^2 
& = & 1\\
\frac{1}{N_l} N_l \sum_{\vec{\delta}\nu^{\prime}\sigma\sigma^{\prime}} 
|\phi_{\nu_{max}\sigma\nu^{\prime}\sigma^{\prime}}(\vec{\delta})|^2 & = & 
1 \\
\sum_{\vec{\delta}\nu^{\prime}\sigma\sigma^{\prime}} 
|\phi_{\nu_{max}\sigma\nu^{\prime}\sigma^{\prime}}(\vec{\delta}) |^2 & = & 1.
\end{eqnarray}
Thus, the final real-space form for $H_p$ is given (in conventional
notation) by
\begin{equation}
\label{H_p_real}
H_p = -\frac{1}{2}
\sum_{\mathbf{r}\mathbf{r}^{\prime}\nu\nu^{\prime}\sigma\sigma^{\prime}} 
 h_p \left(e^{i 2 \pi \vec{n_p}\cdot \vec{r}_{cm}/L} 
\phi_{\nu\sigma\nu^{\prime}\sigma^{\prime}}(\mathbf{r} -
\mathbf{r}^{\prime}) c_{\mathbf{r}\nu\sigma} c_{\mathbf{r}\nu^{\prime}\sigma^{\prime}}
+e^{-i 2 \pi \vec{n_p}\cdot \vec{r}_{cm}/L} 
\phi_{\nu^{\prime}\sigma^{\prime}\nu\sigma}^*(\mathbf{r}^{\prime} -
\mathbf{r})c^{\dagger}_{\mathbf{r}\nu\sigma}
c^{\dagger}_{\mathbf{r}^{\prime}\nu^{\prime}\sigma^{\prime}} \right).
\end{equation}

Next we convert $H_p$ to $\mathbf{k}$-space.  After substituting
using Eq.~(\ref{c-backwards}) we get
\begin{equation}
\begin{split}
H_p = & - \frac{h_p}{2N} \sum_{\mathbf{r}\mathbf{r}^{\prime}\sigma\sigma^{\prime}} 
\sum_{\mathbf{k},\mathbf{k}^{\prime}}
e^{i 2 \pi \vec{n_p}\cdot \vec{r}_{cm}/L}
\phi_{\nu\sigma\nu^{\prime}\sigma^{\prime}}(\mathbf{r} - \mathbf{r}^{\prime})
c_{\mathbf{k}\nu\sigma} c_{\mathbf{k}^{\prime}\nu^{\prime}\sigma^{\prime}}
e^{i \mathbf{k}\cdot\mathbf{r}} e^{i \mathbf{k}^{\prime}\cdot
\mathbf{r}^{\prime}} 
\\
& + e^{-i 2 \pi \vec{n_p}\cdot \vec{r}_{cm}/L} 
\phi_{\nu^{\prime}\sigma^{\prime}\nu\sigma}^*(\mathbf{r}^{\prime} - \mathbf{r})
 c^{\dagger}_{\mathbf{k}^{\prime}\nu\sigma}
 c^{\dagger}_{\mathbf{k}\nu^{\prime}\sigma^{\prime}} 
e^{-i\mathbf{k}^{\prime}\cdot\mathbf{r}}
 e^{-i \mathbf{k}\cdot \mathbf{r}^{\prime}}  \\
 = & -\frac{h_p}{2N} 
 \sum_{\mathbf{k}\mathbf{k}^{\prime}}
 \sum_{\mathbf{r}_{cm} \vec{\delta}\nu\nu^{\prime}\sigma\sigma^{\prime}}
e^{i 2 \pi \vec{n_p}\cdot \vec{r}_{cm}/L} 
\phi_{\nu\sigma\nu^{\prime}\sigma^{\prime}}(\vec{\delta})\,
 c_{\mathbf{k}\nu\sigma} c_{\mathbf{k}^{\prime}\nu^{\prime}\sigma^{\prime}}
 e^{i(\mathbf{k}-\mathbf{k}^{\prime})/\cdot \vec{\delta}/2} 
 e^{i (\mathbf{k}^{\prime}+\mathbf{k})\cdot \mathbf{r}_{cm}}  \\
 & 
 -\frac{h_p}{2N} 
 \sum_{\mathbf{k}\mathbf{k}^{\prime}}
 \sum_{\mathbf{r}_{cm}\vec{\delta}\nu\nu^{\prime}\sigma\sigma^{\prime}} 
  e^{-i 2 \pi \vec{n_p}\cdot \vec{r}_{cm}/L}
\phi_{\nu^{\prime}\sigma^{\prime}\nu\sigma}^*(\vec{\delta})
 c^{\dagger}_{\mathbf{k}^{\prime}\nu\sigma}
 c^{\dagger}_{\mathbf{k}\nu^{\prime}\sigma^{\prime}}
 e^{-i(\mathbf{k}+\mathbf{k}^{\prime})\cdot \mathbf{r}_{cm}}
 e^{-i(\mathbf{k}-\mathbf{k}^{\prime})\cdot\vec{\delta}/2}  \\
= & -\frac{h_p}{2}
\sum_{\mathbf{k}\mathbf{k}^{\prime}\vec{\delta}\nu\nu^{\prime}\sigma\sigma^{\prime}}
\phi_{\nu\sigma\nu^{\prime}\sigma^{\prime}}(\vec{\delta}) c_{\mathbf{k}\nu\sigma}
c_{\mathbf{k}^{\prime}\nu^{\prime}\sigma^{\prime}} 
e^{i(\mathbf{k}-\mathbf{k}^{\prime})\cdot\vec{\delta}}
\delta_{\mathbf{k}^{\prime},-\mathbf{k}- 2 \pi \vec{n}_p /L} \\
& + \phi_{\nu^{\prime}\sigma^{\prime}\nu\sigma}^*(\vec{\delta})
c^{\dagger}_{\mathbf{k}^{\prime}\nu\sigma}
c_{\mathbf{k}\nu^{\prime}\sigma^{\prime}} 
e^{-i (\mathbf{k}-\mathbf{k}^{\prime})\cdot\vec{\delta}}
\delta_{\mathbf{k}^{\prime},-\mathbf{k} - 2\pi \vec{n}_p /L_x}  \\
= & - \frac{h_p}{2} \sum_{\mathbf{k}\vec{\delta}\nu\nu^{\prime}\sigma\sigma^{\prime}}
\phi_{\nu\sigma\nu^{\prime}\sigma^{\prime}}(\vec{\delta})
e^{i (\mathbf{k}+ \pi\vec{n}_p/L)\cdot \vec{\delta}}
c_{\mathbf{k}\nu\sigma} 
c_{-\mathbf{k} - 2 \pi \vec{n}_p / L,\nu^{\prime}\sigma^{\prime}} \\
& + \phi_{\nu^{\prime}\sigma^{\prime}\nu\sigma}^*(\vec{\delta})
e^{-i (\mathbf{k}+ \pi\vec{n}_p/L)\cdot \vec{\delta}}  
c^{\dagger}_{-\mathbf{k} -2 \pi \vec{n}_p / L,\nu\sigma}
c^{\dagger}_{\mathbf{k}\nu^{\prime}\sigma^{\prime}}.
\end{split}
\end{equation}
If we set
\begin{equation}
\phi_{\nu\sigma\nu^{\prime}\sigma^{\prime}}(\mathbf{k}) \equiv \sum_{\vec{\delta}}
e^{-i \mathbf{k}\cdot\vec{\delta}}
\phi_{\nu\sigma\nu^{\prime}\sigma^{\prime}}(\vec{\delta}), 
\end{equation}
then we get
\begin{equation}
H_p = - \frac{h_p}{2} \sum_{\mathbf{k}\nu\nu^{\prime}\sigma\sigma^{\prime}} 
\phi_{\nu\sigma\nu^{\prime}\sigma^{\prime}}(-\mathbf{k}-\pi\vec{n}_p/L)
c_{\mathbf{k}\nu\sigma} 
c_{-\mathbf{k} - 2\pi \vec{n}_p/L,\nu^{\prime}\sigma^{\prime}}
+  \phi_{\nu^{\prime}\sigma^{\prime}\nu\sigma}^*(-\mathbf{k}-\pi\vec{n}_p/L)     
c^{\dagger}_{-\mathbf{k} - 2\pi\vec{n_p}/L\nu\sigma}
c_{\mathbf{k}\nu^{\prime}\sigma^{\prime}}.
\end{equation}

Thus, if $\vec{n}_p \neq 0$, then the electron pairs are
coupled such that their $x$-momenta differ by an amount equal
to $\pi \vec{n}_p/L$.  
For $|\phi_B| < \phi_o / 2$,
we expect $\vec{n}_p$ to be zero.  In that case, we
can write the final form of $H_p$ as
\begin{eqnarray}
H_p & = & - \frac{h_p}{2} \sum_{\mathbf{k}\nu\nu^{\prime}\sigma\sigma^{\prime}} 
\phi_{\nu\sigma\nu^{\prime}\sigma^{\prime}}(-\mathbf{k})
  c_{\mathbf{k}\nu\sigma}
  c_{-\mathbf{k}\nu^{\prime}\sigma^{\prime}}
+ \phi_{\nu^{\prime}\sigma^{\prime}\nu\sigma}^*(-\mathbf{k}) 
  c^{\dagger}_{-\mathbf{k}\nu\sigma}
  c^{\dagger}_{\mathbf{k}\nu^{\prime}\sigma^{\prime}}\\
 & = & - \frac{h_p}{2} \sum_{\mathbf{k}\nu\nu^{\prime}\sigma\sigma^{\prime}} 
\phi_{\nu\sigma\nu^{\prime}\sigma^{\prime}}(-\mathbf{k})
  c_{\mathbf{k}\nu\sigma}
  c_{-\mathbf{k}\nu^{\prime}\sigma^{\prime}}
+ \phi_{\nu^{\prime}\sigma^{\prime}\nu\sigma}^*(\mathbf{k}) 
  c^{\dagger}_{\mathbf{k}\nu\sigma}
  c^{\dagger}_{-\mathbf{k}\nu^{\prime}\sigma^{\prime}}
\end{eqnarray}


