\chapter{Nambu formalism and the non-interacting Green's function}
\label{chapter:g0}

\section{Definitions}

To account for broken U(1) symmetry in the Green's function, we
introduce the Nambu representation for the field operator, $\psi$, as
\begin{eqnarray}
\psi_{\mathbf{r}\nu 0} & = & c_{\mathbf{r}\nu\uparrow} \\
\psi_{\mathbf{r}\nu 1} & = & c_{\mathbf{r}\nu\downarrow} \\
\psi_{\mathbf{r}\nu 2} & = & c^{\dagger}_{\mathbf{r}\nu\uparrow} \\
\psi_{\mathbf{r}\nu 3} & = & c^{\dagger}_{\mathbf{r}\nu\downarrow} 
\end{eqnarray}
with Hermetian conjugates
\begin{eqnarray}
\psi^{\dagger}_{\mathbf{r}\nu 0} & = & c^{\dagger}_{\mathbf{r}\nu \uparrow} \\
\psi^{\dagger}_{\mathbf{r}\nu 1} & = & c^{\dagger}_{\mathbf{r}\nu \downarrow} \\
\psi^{\dagger}_{\mathbf{r}\nu 2} & = & c_{\mathbf{r}\nu \uparrow} \\
\psi^{\dagger}_{\mathbf{r}\nu 3} & = & c_{\mathbf{r}\nu \downarrow}. 
\end{eqnarray}
We treat these four components as corresponding to
four distinct particle types.  Of course, this
effectively double counts the number of degrees of freedom
and this must be accounted for in expressions for the
self-energy and
thermodynamic functions.

We define the momentum space operators via
\begin{equation}
\psi_{\mathbf{k}\nu\alpha} \equiv
\frac{1}{\sqrt{N}} 
\sum_{\mathbf{r}}
e^{-i \mathbf{k}\cdot \mathbf{r}} \psi_{\mathbf{r}\nu \alpha}
\end{equation}
so that
\begin{eqnarray}
\psi_{\mathbf{k}\nu 0} & = & c_{\mathbf{k}\nu\uparrow} \\
\psi_{\mathbf{k}\nu 1} & = & c_{\mathbf{k}\nu\downarrow} \\
\psi_{\mathbf{k}\nu 2} & = & c^{\dagger}_{-\mathbf{k}\nu\uparrow} \\
\psi_{\mathbf{k}\nu 3} & = & c^{\dagger}_{-\mathbf{k}\nu\downarrow}
\end{eqnarray}

The single-particle Green's function is defined as
\begin{equation}
G_{\nu\alpha\nu^{\prime}\alpha^{\prime}}(\mathbf{k},\tau) \equiv
- \cal{T}_{\tau} <\psi_{\mathbf{k}\nu\alpha}(\tau) 
\psi^{\dagger}_{\mathbf{k}\nu^{\prime}\alpha^{\prime}}>
\end{equation}
Although the momentum indices could be distinct in the
above, the Hamiltonian contains no terms that do not
obey momentum conservation when the Hamiltonian is
expressed properly in terms of $\psi$ and $\psi^{\dagger}$.
Thus, only the momentum diagonal form shown above is
non-zero.

\section{Evaluation of $G^{(0)}$}

We utilize the equation of motion technique to evaluate
the non-interacting Green's function, $G^{(0)}$.
In all cases, we have
\begin{equation}
\psi_{\mathbf{k}\nu\alpha}(\tau) \equiv
e^{(\hat{H}-\mu\hat{N})\tau} \psi_{\mathbf{k}\nu\alpha}
e^{-(\hat{H}-\mu\hat{N})\tau} 
\end{equation}
so that
\begin{eqnarray}
\frac{\partial \psi_{\mathbf{k}\nu\alpha}(\tau)}{\partial \tau} & = &
(\hat{H} - \mu \hat{N}) \psi_{\mathbf{k}\nu\alpha}(\tau) -
 \psi_{\alpha,\mathbf{k}}(\tau)(\hat{H} - \mu \hat{N}) \\
& = & \left[ (\hat{H} - \mu\hat{N}),\psi_{\mathbf{k}\nu\alpha}(\tau) \right]
\end{eqnarray}
In the absence of interactions we have
\begin{equation}
\begin{split}
\hat{H}^{(0)} - \mu \hat{N} = & \sum_{\mathbf{k}^{\prime}\nu^{\prime}\nu^{\prime\prime}\sigma^{\prime}}
 \xi_{\nu^{\prime}\nu^{\prime\prime}}(\mathbf{k}) 
c^{\dagger}_{\mathbf{k}^{\prime}\nu^{\prime}\sigma^{\prime}}
c_{\mathbf{k}^{\prime}\nu^{\prime\prime}\sigma^{\prime}} -
\sum_{\mathbf{k}^{\prime}\nu^{\prime}\sigma^{\prime}\sigma^{\prime\prime}}
\mathbf{h}\cdot\vec{\sigma}_{\sigma^{\prime}\sigma^{\prime\prime}}
c^{\dagger}_{\mathbf{k}^{\prime}\nu^{\prime}\sigma^{\prime}}
c_{\mathbf{k}^{\prime}\nu^{\prime}\sigma^{\prime\prime}} \\
& - \frac{h_p}{2}
\sum_{\mathbf{k}^{\prime}\nu^{\prime}\nu^{\prime\prime}\sigma^{\prime}\sigma^{\prime\prime}}
\left(\phi_{\nu^{\prime}\sigma^{\prime}\nu^{\prime\prime}\sigma^{\prime\prime}}(-\mathbf{k}^{\prime})
c_{\mathbf{k}^{\prime}\nu^{\prime}\sigma^{\prime}}
c_{-\mathbf{k}^{\prime}\nu^{\prime\prime}\sigma^{\prime\prime}}
+
\phi^*_{\nu^{\prime\prime}\sigma^{\prime\prime}\nu^{\prime}\sigma^{\prime}}(\mathbf{k}^{\prime})
c^{\dagger}_{\mathbf{k}^{\prime}\nu^{\prime}\sigma^{\prime}}
c^{\dagger}_{-\mathbf{k}^{\prime}\nu^{\prime\prime}\sigma^{\prime\prime}}\right)
\end{split}
\end{equation}
and these commutation relations are simple to evaluate.

For $\alpha = 0$ or 1 we have
\begin{equation}
\psi_{\mathbf{k}\nu\alpha}(\tau) = c_{\mathbf{k}\nu\sigma}(\tau).
\end{equation}
When we commute this with the hopping (and chemical potential) term in the
Hamiltonian we get
\begin{eqnarray}
& & 
\sum_{\mathbf{k}^{\prime}\nu^{\prime}\nu^{\prime\prime}\sigma^{\prime}} 
\xi_{\nu^{\prime}\nu^{\prime\prime}}(\mathbf{k}^{\prime})
\left[c^{\dagger}_{\mathbf{k}^{\prime}\nu^{\prime}\sigma^{\prime}}
c_{\mathbf{k}^{\prime}\nu^{\prime\prime}\sigma^{\prime}},
c_{\mathbf{k}\nu\sigma} \right] \\
& = & 
\sum_{\mathbf{k}^{\prime}\nu^{\prime}\nu^{\prime\prime}\sigma^{\prime}} 
\xi_{\nu^{\prime}\nu^{\prime\prime}}(\mathbf{k}^{\prime})
\left(c^{\dagger}_{\mathbf{k}^{\prime}\nu^{\prime}\sigma^{\prime}}
c_{\mathbf{k}^{\prime}\nu^{\prime\prime}\sigma^{\prime}}
c_{\mathbf{k}\nu\sigma} -
c_{\mathbf{k}\nu\sigma}
c^{\dagger}_{\mathbf{k}^{\prime}\nu^{\prime}\sigma^{\prime}}
c_{\mathbf{k}^{\prime}\nu^{\prime\prime}\sigma^{\prime}} \right)
\\
& = & 
\sum_{\mathbf{k}^{\prime}\nu^{\prime}\nu^{\prime\prime}\sigma^{\prime}} 
\xi_{\nu^{\prime}\nu^{\prime\prime}}(\mathbf{k}^{\prime})
\left(-c^{\dagger}_{\mathbf{k}^{\prime}\nu^{\prime}\sigma^{\prime}}
c_{\mathbf{k}\nu\sigma}
c_{\mathbf{k}^{\prime}\nu^{\prime\prime}\sigma^{\prime}} -
c_{\mathbf{k}\nu\sigma}
c^{\dagger}_{\mathbf{k}^{\prime}\nu^{\prime}\sigma^{\prime}}
c_{\mathbf{k}^{\prime}\nu^{\prime\prime}\sigma^{\prime}}
 \right) \\
& = & \sum_{\mathbf{k}^{\prime}\nu^{\prime}\nu^{\prime\prime}\sigma^{\prime}} 
\xi_{\nu^{\prime}\nu^{\prime\prime}}(\mathbf{k}^{\prime})
\left(-\delta_{\mathbf{k}^{\prime},\mathbf{k}}\delta_{\nu^{\prime}\nu}
\delta_{\sigma,\sigma^{\prime}}c_{\mathbf{k}^{\prime}\nu^{\prime\prime}\sigma^{\prime}}
\right) 
\\
& = & - \sum_{\nu^{\prime\prime}}\xi_{\nu\nu^{\prime\prime}}(\mathbf{k}) 
c_{\mathbf{k}\nu^{\prime\prime}\sigma}
\end{eqnarray}

The magnetic field term yields
\begin{eqnarray}
& & -\sum_{\mathbf{k}^{\prime}\nu^{\prime}\sigma^{\prime}
\sigma^{\prime\prime}}
\mathbf{h}\cdot \vec{\sigma}_{\sigma^{\prime}\sigma^{\prime\prime}}
\left[ c^{\dagger}_{\mathbf{k}^{\prime}\nu^{\prime}\sigma^{\prime}}
       c_{\mathbf{k}^{\prime}\nu^{\prime}\sigma^{\prime\prime}},
       c_{\mathbf{k}\nu\sigma} \right] \\
& &= -\sum_{\mathbf{k}^{\prime}\nu^{\prime}\sigma^{\prime}
\sigma^{\prime\prime}}
\mathbf{h}\cdot \vec{\sigma}_{\sigma^{\prime}\sigma^{\prime\prime}}
\left( c^{\dagger}_{\mathbf{k}^{\prime}\nu^{\prime}\sigma^{\prime}}
       c_{\mathbf{k}^{\prime}\nu^{\prime}\sigma^{\prime\prime}}
       c_{\mathbf{k}\nu\sigma} -
       c_{\mathbf{k}\nu\sigma}
 c^{\dagger}_{\mathbf{k}^{\prime}\nu^{\prime}\sigma^{\prime}}
       c_{\mathbf{k}^{\prime}\nu^{\prime}\sigma^{\prime\prime}}\right) \\
& &= -\sum_{\mathbf{k}^{\prime}\nu^{\prime}\sigma^{\prime}
\sigma^{\prime\prime}}
\mathbf{h}\cdot \vec{\sigma}_{\sigma^{\prime}\sigma^{\prime\prime}}
\left(- c^{\dagger}_{\mathbf{k}^{\prime}\nu^{\prime}\sigma^{\prime}} 
 c_{\mathbf{k}\nu\sigma} 
c_{\mathbf{k}^{\prime}\nu^{\prime}\sigma^{\prime}} -
 c_{\mathbf{k}\nu\sigma}
 c^{\dagger}_{\mathbf{k}^{\prime}\nu^{\prime}\sigma^{\prime}}
       c_{\mathbf{k}^{\prime}\nu^{\prime}\sigma^{\prime\prime}}\right) \\
& & = -\sum_{\mathbf{k}^{\prime}\nu^{\prime}\sigma^{\prime}
\sigma^{\prime\prime}}
\mathbf{h}\cdot \vec{\sigma}_{\sigma^{\prime}\sigma^{\prime\prime}}
\left( -\delta_{\sigma^{\prime},\sigma} 
\delta_{\mathbf{k},\mathbf{k}^{\prime}}\delta_{\nu,\nu^{\prime}}
c_{\mathbf{k}^{\prime}\nu^{\prime}\sigma^{\prime\prime}} \right) \\
& & = \sum_{\sigma^{\prime\prime}}\mathbf{h}\cdot
\vec{\sigma}_{\sigma\sigma^{\prime\prime}}\,c_{\mathbf{k}\nu\sigma^{\prime\prime}}
\end{eqnarray}

The off-diagonal terms give us
\begin{eqnarray}
\nonumber
& & -\frac{h_p}{2}
\sum_{\mathbf{k}^{\prime}\nu^{\prime}\nu^{\prime\prime}\sigma^{\prime}\sigma^{\prime\prime}}
\left(\phi_{\nu^{\prime}\sigma^{\prime}\nu^{\prime\prime}\sigma^{\prime\prime}}(-\mathbf{k}^{\prime})
\left[ c_{\mathbf{k}^{\prime}\nu^{\prime}\sigma^{\prime}}
c_{-\mathbf{k}^{\prime}\nu^{\prime\prime}\sigma^{\prime\prime}},
c_{\mathbf{k}\nu\sigma}\right]
+
\phi^*_{\nu^{\prime\prime}\sigma^{\prime\prime}\nu^{\prime}\sigma^{\prime}}(\mathbf{k}^{\prime})
\left[c^{\dagger}_{\mathbf{k}^{\prime}\nu^{\prime}\sigma^{\prime}}
c^{\dagger}_{-\mathbf{k}^{\prime}\nu^{\prime\prime}\sigma^{\prime\prime},},
c_{\mathbf{k}\nu\sigma}\right]
\right) \\
& = &
- \frac{h_p}{2}
\sum_{\mathbf{k}^{\prime}\nu^{\prime}\nu^{\prime\prime}\sigma^{\prime}\sigma^{\prime\prime}}
\phi^*_{\nu^{\prime\prime}\sigma^{\prime\prime}\nu^{\prime}\sigma^{\prime}}(\mathbf{k}^{\prime})
\left(
c^{\dagger}_{\mathbf{k}^{\prime}\nu^{\prime}\sigma^{\prime}}
c^{\dagger}_{-\mathbf{k}^{\prime}\nu^{\prime\prime}\sigma^{\prime\prime}}
c_{\mathbf{k}\nu\sigma}-
c_{\mathbf{k}\nu\sigma}
c^{\dagger}_{\mathbf{k}^{\prime}\nu^{\prime}\sigma^{\prime}}
c^{\dagger}_{-\mathbf{k}^{\prime}\nu^{\prime\prime}\sigma^{\prime\prime}}
\right) \\
\nonumber
& = &
- \frac{h_p}{2}
\sum_{\mathbf{k}^{\prime}\nu^{\prime}\nu^{\prime\prime}\sigma^{\prime}\sigma^{\prime\prime}}
\phi^*_{\nu^{\prime\prime}\sigma^{\prime\prime}\nu^{\prime}\sigma^{\prime}}(\mathbf{k}^{\prime})
\left(c^{\dagger}_{\mathbf{k}^{\prime}\nu^{\prime}\sigma^{\prime}}
\delta_{\mathbf{k},-\mathbf{k}^{\prime}}\delta_{\nu,\nu^{\prime\prime}}
\delta_{\sigma\sigma^{\prime\prime}}
-c^{\dagger}_{\mathbf{k}^{\prime}\nu^{\prime}\sigma^{\prime}}
c_{\mathbf{k}\nu\sigma}
c^{\dagger}_{-\mathbf{k}^{\prime}\nu^{\prime\prime}\sigma^{\prime\prime}}
-c_{\mathbf{k}\nu\sigma}
c^{\dagger}_{\mathbf{k}^{\prime}\nu^{\prime}\sigma^{\prime}}
c^{\dagger}_{-\mathbf{k}^{\prime}\nu^{\prime\prime}\sigma^{\prime\prime}}\right) \\
& = &
- \frac{h_p}{2}
\sum_{\mathbf{k}^{\prime}\nu^{\prime}\nu^{\prime\prime}\sigma^{\prime}\sigma^{\prime\prime}}
\phi^*_{\nu^{\prime\prime}\sigma^{\prime\prime}\nu^{\prime}\sigma^{\prime}}(\mathbf{k}^{\prime})
\left(c^{\dagger}_{\mathbf{k}^{\prime}\nu^{\prime}\sigma^{\prime}}
\delta_{\mathbf{k},-\mathbf{k}^{\prime}}\delta_{\nu\nu^{\prime\prime}}
\delta_{\sigma\sigma^{\prime\prime}}
-c^{\dagger}_{-\mathbf{k}^{\prime}\nu^{\prime\prime}\sigma^{\prime\prime}}
\delta_{\mathbf{k}^{\prime},\mathbf{k}}\delta_{\nu,\nu^{\prime}}
\delta_{\sigma\sigma^{\prime}}
\right) \\
& = & - \frac{h_p}{2} \left(
\sum_{\nu^{\prime}\sigma^{\prime}} 
\phi^*_{\nu\sigma\nu^{\prime}\sigma^{\prime}}(-\mathbf{k}) 
c^{\dagger}_{-\mathbf{k}\nu^{\prime}\sigma^{\prime}}
-\sum_{\nu^{\prime\prime}\sigma^{\prime\prime}} 
\phi^*_{\nu^{\prime\prime}\sigma^{\prime\prime}\nu\sigma}(\mathbf{k})
c^{\dagger}_{-\mathbf{k}\nu^{\prime\prime}\sigma^{\prime\prime}} \right) \\
& = &
- \frac{h_p}{2}\sum_{\nu^{\prime}\sigma^{\prime}} 
\left(\phi^*_{\nu\sigma\nu^{\prime}\sigma^{\prime}}(-\mathbf{k})
- \phi^*_{\nu^{\prime}\sigma^{\prime}\nu\sigma}(\mathbf{k}) \right)
c^{\dagger}_{-\mathbf{k}\nu^{\prime}\sigma^{\prime}}
\\
& = & -h_p \sum_{\nu^{\prime}\sigma^{\prime}} 
\phi^*_{\nu\sigma\nu^{\prime}\sigma^{\prime}}(-\mathbf{k})\,
c^{\dagger}_{-\mathbf{k}\nu^{\prime}\sigma^{\prime}}
\end{eqnarray}

Consequently, for $\alpha = 0$ and 1 we have
\begin{equation}
\frac{\partial \psi_{\mathbf{k}\nu\alpha}(\tau)}{\partial \tau}
 = \sum_{\nu^{\prime}\alpha^{\prime}} 
B_{\nu\alpha\nu^{\prime}\alpha^{\prime}}(\mathbf{k})\,
\psi_{\mathbf{k}\nu^{\prime}\alpha^{\prime}}(\tau)
\end{equation}
where
\begin{eqnarray}
B_{\nu 0 \nu^{\prime} 0} & = & -\xi_{\nu\nu^{\prime}}(\mathbf{k}) + 
\mathbf{h}\cdot\sigma_{00}\delta_{\nu,\nu^{\prime}} =
 -\xi_{\nu\nu^{\prime}}(\mathbf{k}) + h_z \delta_{\nu\nu^{\prime}} \\
B_{\nu 0 \nu^{\prime}1} & = & \mathbf{h}\cdot\sigma_{01}\delta_{\nu\nu^{\prime}} = 
\left(h_x - i h_y\right)\delta_{\nu\nu^{\prime}} \\
B_{\nu 0 \nu^{\prime} 2} & = & 
-h_p\, \phi^*_{\nu\uparrow\nu^{\prime}\uparrow}(-\mathbf{k}) \\
B_{\nu 0 \nu^{\prime} 3} & = & -h_p\, \phi^*_{\nu\uparrow\nu^{\prime}\downarrow}(-\mathbf{k}) \\
B_{\nu 1 \nu^{\prime} 0} & = & \mathbf{h}\cdot\sigma_{10}\delta_{\nu\nu^{\prime}} = 
\left(h_x + i h_y\right)\delta_{\nu\nu^{\prime}} \\
B_{\nu 1 \nu^{\prime}1} & = & -\xi_{\nu\nu^{\prime}}(\mathbf{k}) + 
\mathbf{h}\cdot\sigma_{11}\delta_{\nu\nu^{\prime}} =
 -\xi_{\nu\nu^{\prime}}(\mathbf{k}) - h_z \delta_{\nu\nu^{\prime}} \\
B_{\nu 1 \nu^{\prime}2} & = &  -h_p\, \phi^*_{\nu\downarrow\nu^{\prime}\uparrow}(-\mathbf{k}) \\
B_{\nu 1 \nu^{\prime} 3} & = & -h_p\, 
\phi^*_{\nu\downarrow\nu^{\prime}\downarrow}(-\mathbf{k}) 
\end{eqnarray}

If $\alpha = 2$ or 3, then
\begin{equation}
\psi_{\mathbf{k}\nu\alpha}(\tau) = c^{\dagger}_{-\mathbf{k}\nu\sigma}(\tau).
\end{equation}
Taking the commutator with the normal terms in the
non-interacting Hamiltonian yields
\begin{eqnarray}
& & \sum_{\mathbf{k}^{\prime}\nu^{\prime}\nu^{\prime\prime}\sigma^{\prime}}
\xi_{\nu^{\prime}\nu^{\prime\prime}}(\mathbf{k}^{\prime})
\left[ c^{\dagger}_{\mathbf{k}^{\prime}\nu^{\prime}\sigma^{\prime}}
c_{\mathbf{k}^{\prime}\nu^{\prime\prime}\sigma^{\prime}},
c^{\dagger}_{-\mathbf{k}\nu\sigma}
\right] \\
& = & 
\sum_{\mathbf{k}^{\prime}\nu^{\prime}\nu^{\prime\prime}\sigma^{\prime}}
\xi_{\nu^{\prime}\nu^{\prime\prime}}(\mathbf{k}^{\prime})
\left(c^{\dagger}_{\mathbf{k}^{\prime}\nu^{\prime}\sigma^{\prime}}
c_{\mathbf{k}^{\prime}\nu^{\prime\prime}\sigma^{\prime}}
c^{\dagger}_{-\mathbf{k}\nu\sigma} -
c^{\dagger}_{-\mathbf{k}\nu\sigma}
c^{\dagger}_{\mathbf{k}^{\prime}\nu^{\prime}\sigma^{\prime}}
c_{\mathbf{k}^{\prime}\nu^{\prime\prime}\sigma^{\prime}}
\right) \\
& = &
\sum_{\mathbf{k}^{\prime}\nu^{\prime}\nu^{\prime\prime}\sigma^{\prime}}
\xi_{\nu^{\prime}\nu^{\prime\prime}}(\mathbf{k}^{\prime})
\left(c^{\dagger}_{\mathbf{k}^{\prime}\nu^{\prime}\sigma^{\prime}}
\delta_{\mathbf{k}^{\prime},-\mathbf{k}}\delta_{\nu\nu^{\prime\prime}}
\delta_{\sigma^{\prime},\sigma} -
c^{\dagger}_{\mathbf{k}^{\prime}\nu^{\prime}\sigma^{\prime}}
c^{\dagger}_{-\mathbf{k}\nu\sigma}
c_{\mathbf{k}^{\prime}\nu^{\prime\prime}\sigma^{\prime}} -
c^{\dagger}_{-\mathbf{k}\nu\sigma}
c^{\dagger}_{\mathbf{k}^{\prime}\nu^{\prime}\sigma^{\prime}}
c_{\mathbf{k}^{\prime}\nu^{\prime\prime}\sigma^{\prime}} \right)
\\
& = &
\sum_{\mathbf{k}^{\prime}\nu^{\prime}\nu^{\prime\prime}\sigma^{\prime}}
\xi_{\nu^{\prime}\nu^{\prime\prime}}(\mathbf{k}^{\prime})
c^{\dagger}_{\mathbf{k}^{\prime}\nu^{\prime}\sigma^{\prime}}
\delta_{\mathbf{k}^{\prime},-\mathbf{k}}\delta_{\nu,\nu^{\prime\prime}}
\delta_{\sigma^{\prime},\sigma} \\
& = & \sum_{\nu^{\prime}}
\xi_{\nu^{\prime}\nu}(-\mathbf{k})\,c^{\dagger}_{-\mathbf{k}\nu^{\prime}\sigma}
\end{eqnarray}

The magnetic field term produces the following
\begin{eqnarray}
& & -\sum_{\mathbf{k}^{\prime}\nu^{\prime}\sigma^{\prime}\sigma^{\prime\prime}}
\mathbf{h}\cdot \vec{\sigma}_{\sigma^{\prime}\sigma^{\prime\prime}}
\left[c^{\dagger}_{\mathbf{k}^{\prime}\nu^{\prime}\sigma^{\prime}}
c_{\mathbf{k}^{\prime}\nu^{\prime}\sigma^{\prime\prime}},
c^{\dagger}_{-\mathbf{k}\nu\sigma}\right] \\
& = &  -\sum_{\mathbf{k}^{\prime}\nu^{\prime}\sigma^{\prime}\sigma^{\prime\prime}}
\mathbf{h}\cdot \vec{\sigma}_{\sigma^{\prime}\sigma^{\prime\prime}}
\left( c^{\dagger}_{\mathbf{k}^{\prime}\nu^{\prime}\sigma^{\prime}}
c_{\mathbf{k}^{\prime}\nu^{\prime}\sigma^{\prime\prime}}
c^{\dagger}_{-\mathbf{k}\nu\sigma} -
c^{\dagger}_{-\mathbf{k}\nu\sigma}
c^{\dagger}_{\mathbf{k}^{\prime}\nu^{\prime}\sigma^{\prime}}
c_{\mathbf{k}^{\prime}\nu^{\prime}\sigma^{\prime\prime}} \right) \\
& = & -\sum_{\mathbf{k}^{\prime}\nu^{\prime}\sigma^{\prime}\sigma^{\prime\prime}}
\mathbf{h}\cdot \vec{\sigma}_{\sigma^{\prime}\sigma^{\prime\prime}}
\left( c^{\dagger}_{\mathbf{k}^{\prime}\nu^{\prime}\sigma^{\prime}}
c_{\mathbf{k}^{\prime}\nu^{\prime}\sigma^{\prime\prime}}
c^{\dagger}_{-\mathbf{k}\nu\sigma} + 
c^{\dagger}_{\mathbf{k}^{\prime}\nu^{\prime}\sigma^{\prime}}
c^{\dagger}_{-\mathbf{k}\nu\sigma}
c_{\mathbf{k}^{\prime}\nu^{\prime}\sigma^{\prime\prime}} \right) \\
& = & -\sum_{\mathbf{k}^{\prime}\nu^{\prime}\sigma^{\prime}\sigma^{\prime\prime}}
\mathbf{h}\cdot \vec{\sigma}_{\sigma^{\prime}\sigma^{\prime\prime}}
\left( c^{\dagger}_{\mathbf{k}^{\prime}\nu^{\prime}\sigma^{\prime}}
\delta_{\sigma\sigma^{\prime\prime}} \delta_{-\mathbf{k},
\mathbf{k}^{\prime}}\delta_{\nu\nu^{\prime}} \right) \\
& = & - \sum_{\sigma^{\prime}} \mathbf{h}\cdot
\vec{\sigma}_{\sigma^{\prime}\sigma}
c^{\dagger}_{-\mathbf{k}\nu\sigma^{\prime}}
\end{eqnarray}

The anomalous terms yield
\begin{eqnarray}
& & -\frac{h_p}{2}
\sum_{\mathbf{k}^{\prime}\nu^{\prime}\nu^{\prime\prime}\sigma^{\prime}\sigma^{\prime\prime}}
\left(\phi_{\nu^{\prime}\sigma^{\prime}\nu^{\prime\prime}\sigma^{\prime\prime}}(-\mathbf{k})
\left[ c_{\mathbf{k}^{\prime}\nu^{\prime}\sigma^{\prime}}
c_{-\mathbf{k}^{\prime}\nu^{\prime\prime}\sigma^{\prime\prime}},
c^{\dagger}_{-\mathbf{k}\nu\sigma}\right]
+ \phi^*_{\nu^{\prime\prime}\sigma^{\prime\prime}\nu^{\prime}\sigma^{\prime}}(\mathbf{k}^{\prime})
\left[ c^{\dagger}_{\mathbf{k}^{\prime}\nu^{\prime}\sigma^{\prime}}
c^{\dagger}_{-\mathbf{k}\nu^{\prime\prime}\sigma^{\prime\prime}},
c^{\dagger}_{-\mathbf{k}\nu\sigma} \right] \right) \\
& = & -\frac{h_p}{2}
\sum_{\mathbf{k}^{\prime}\nu^{\prime}\nu^{\prime\prime}\sigma^{\prime}\sigma^{\prime\prime}}
\phi_{\nu^{\prime}\sigma^{\prime}\nu^{\prime\prime}\sigma^{\prime\prime}}(-\mathbf{k})
\left(c_{\mathbf{k}^{\prime}\nu^{\prime}\sigma^{\prime}}
c_{-\mathbf{k}^{\prime}\nu^{\prime\prime}\sigma^{\prime\prime}}
c^{\dagger}_{-\mathbf{k}\nu\sigma} -
c^{\dagger}_{-\mathbf{k}\nu\sigma}
c_{\mathbf{k}^{\prime}\nu^{\prime}\sigma^{\prime}}
c_{-\mathbf{k}^{\prime}\nu^{\prime\prime}\sigma^{\prime\prime}} \right)\\
& = & -\frac{h_p}{2}
\sum_{\mathbf{k}^{\prime}\nu^{\prime\prime}\sigma^{\prime}\sigma^{\prime\prime}}
\phi_{\nu^{\prime}\sigma^{\prime}\nu^{\prime\prime}\sigma^{\prime\prime}}(-\mathbf{k})
\left(c_{\mathbf{k}^{\prime}\nu^{\prime}\sigma^{\prime}} 
\delta_{\mathbf{k},\mathbf{k}^{\prime}}\delta_{\nu\nu^{\prime\prime}}
\delta_{\sigma,\sigma^{\prime\prime}}
-
c_{\mathbf{k}^{\prime}\nu^{\prime}\sigma^{\prime}}
c^{\dagger}_{-\mathbf{k}\nu\sigma}
c_{-\mathbf{k}^{\prime}\nu^{\prime\prime}\sigma^{\prime\prime}}
-
c^{\dagger}_{-\mathbf{k}\nu\sigma}
c_{\mathbf{k}^{\prime}\nu^{\prime}\sigma^{\prime}}
c_{-\mathbf{k}^{\prime}\nu^{\prime\prime}\sigma^{\prime\prime}} \right) \\
& = & -\frac{h_p}{2}
\sum_{\mathbf{k}^{\prime}\nu^{\prime}\nu^{\prime\prime}\sigma^{\prime}\sigma^{\prime\prime}}
\phi_{\nu^{\prime}\sigma^{\prime}\nu^{\prime\prime}\sigma^{\prime\prime}}(-\mathbf{k})
\left(c_{\mathbf{k}^{\prime}\nu^{\prime}\sigma^{\prime}} 
\delta_{\mathbf{k},\mathbf{k}^{\prime}}
\delta_{\sigma,\sigma^{\prime\prime}}\delta_{\nu\nu^{\prime\prime}}
- \delta_{\mathbf{k},-\mathbf{k}^{\prime}}
\delta_{\sigma,\sigma^{\prime}}\delta_{\nu\nu^{\prime}} 
c_{-\mathbf{k}^{\prime}\nu^{\prime\prime}\sigma^{\prime\prime}} \right) \\
& = & -\frac{h_p}{2}
\left(
\sum_{\sigma^{\prime}\nu^{\prime}} \phi_{\nu^{\prime}\sigma^{\prime}\nu\sigma}(-\mathbf{k})
c_{\mathbf{k}\nu^{\prime}\sigma^{\prime}} -
\sum_{\nu^{\prime\prime}\sigma^{\prime\prime}}
\phi_{\nu\sigma\nu^{\prime\prime}\sigma^{\prime\prime}}(\mathbf{k})
c_{\mathbf{k}\nu^{\prime\prime}\sigma^{\prime\prime}} \right) \\
& = & -\frac{h_p}{2}
\sum_{\nu^{\prime}\sigma^{\prime}}
\left(\phi_{\nu^{\prime}\sigma^{\prime}\nu\sigma}(-\mathbf{k})
c_{\mathbf{k}\nu^{\prime}\sigma^{\prime}} -
\phi_{\nu\sigma\nu^{\prime}\sigma^{\prime}}(\mathbf{k})
c_{\mathbf{k}\nu^{\prime}\sigma^{\prime}} \right) \\
& = &  -h_p
\sum_{\nu^{\prime}\sigma^{\prime}}
\phi_{\nu^{\prime}\sigma^{\prime}\nu\sigma}(-\mathbf{k})\,
c_{\mathbf{k}\nu^{\prime}\sigma^{\prime}}
\end{eqnarray}

Consequently, for $\alpha = 2$ and 3 we have
\begin{equation}
\frac{\partial \psi_{\mathbf{k}\nu\alpha}(\tau)}{\partial \tau}
 = \sum_{\nu^{\prime}\alpha^{\prime}}
B_{\nu\alpha\nu^{\prime}\alpha^{\prime}}(\mathbf{k})\,
\psi_{\mathbf{k}\nu^{\prime}\alpha^{\prime}}(\tau)
\end{equation}
where
\begin{eqnarray}
B_{\nu 2 \nu^{\prime} 0} & = & -h_p\, 
\phi_{\nu^{\prime}\uparrow\nu\uparrow}(-\mathbf{k}) \\
B_{\nu 2 \nu^{\prime} 1} & = & -h_p\, 
\phi_{\nu^{\prime}\downarrow\nu\uparrow}(-\mathbf{k}) \\
B_{\nu 2 \nu^{\prime}2} & = & \xi_{\nu^{\prime}\nu}(-\mathbf{k}) -
\mathbf{h}\cdot\mathbf{\sigma}_{00}\delta_{\nu\nu^{\prime}}
= \xi_{\nu^{\prime}\nu}(-\mathbf{k}) - h_z\delta_{\nu\nu^{\prime}} \\
B_{\nu 2 \nu^{\prime}3} & = &  -\mathbf{h}\cdot\vec{\sigma}_{10}
\delta_{\nu\nu^{\prime}}
= \left(-h_x -ih_y\right)\delta_{\nu\nu^{\prime}} \\
B_{\nu 3 \nu^{\prime}0} & = &  -h_p\, \phi_{\nu^{\prime}\uparrow\nu\downarrow}(-\mathbf{k}) \\
B_{\nu 3 \nu^{\prime}1} & = & -h_p\, 
\phi_{\nu^{\prime}\downarrow\nu\downarrow}(-\mathbf{k}) \\ 
B_{\nu 3 \nu^{\prime}2} & = & -\mathbf{h}\cdot\mathbf{\sigma}_{01}
\delta_{\nu\nu^{\prime}} =
\left(-h_x + ih_y\right)\delta_{\nu\nu^{\prime}} \\
B_{\nu 3 \nu^{\prime}3} & = & 
\xi_{\nu^{\prime}\nu}(-\mathbf{k}) -
\mathbf{h}\cdot\mathbf{\sigma}_{11}\delta_{\nu\nu^{\prime}}
= \xi_{\nu^{\prime}\nu}(-\mathbf{k}) + h_z\delta_{\nu\nu^{\prime}}  \\
\end{eqnarray}

We use these relations to determine the Fourier transform
of the non-interacting Green's function.  By definition we
have
\begin{equation}
G_{\nu\alpha\nu^{\prime}\alpha^{\prime}}(\epsilon_n,\mathbf{k}) \equiv
\int^{\beta}_0 d\tau\,e^{i\epsilon_n \tau} 
G_{\nu\alpha\nu^{\prime}\alpha^{\prime}}(\tau,\mathbf{k})
\end{equation}
and using integration by parts we get
\begin{eqnarray}
G_{\nu\alpha\nu^{\prime}\alpha^{\prime}}(\epsilon_n,\mathbf{k}) & = &
\frac{1}{i\varepsilon_n} e^{i\varepsilon_n\tau} 
G_{\nu\alpha\nu^{\prime}\alpha^{\prime}}(\tau,\mathbf{k}) |^{\beta}_0
- \frac{1}{i\varepsilon_n} 
\int^{\beta}_0 d\tau\,e^{i\epsilon_n \tau}
\frac{\partial}{\partial\tau} G_{\nu\alpha\nu^{\prime}\alpha^{\prime}}(\tau,\mathbf{k}) \\
& = &
\frac{1}{i\varepsilon_n}
\left(-G_{\nu\alpha\nu^{\prime}\alpha^{\prime}}(\beta^-,\mathbf{k}) -
       G_{\nu\alpha\nu^{\prime}\alpha^{\prime}}(0^+,\mathbf{k}) \right)
- \frac{1}{i\varepsilon_n} 
\int^{\beta}_0 d\tau\,e^{i\epsilon_n \tau}
\frac{\partial}{\partial\tau} 
G_{\nu\alpha\nu^{\prime}\alpha^{\prime}}(\tau,\mathbf{k}) \\
& = &
\frac{1}{i\varepsilon_n}
\left(G_{\nu\alpha\nu^{\prime}\alpha^{\prime}}(0^-,\mathbf{k}) -
       G_{\nu\alpha\nu^{\prime}\alpha^{\prime}}(0^+,\mathbf{k}) \right)
- \frac{1}{i\varepsilon_n} 
\int^{\beta}_0 d\tau\,e^{i\epsilon_n \tau}
\frac{\partial}{\partial\tau} G_{\nu\alpha\nu^{\prime}\alpha^{\prime}}(\tau,\mathbf{k}) \\
& = &
\frac{\delta_{\alpha\alpha^{\prime}}\delta_{\nu\nu^{\prime}}}{i\varepsilon_n}
- \frac{1}{i\varepsilon_n} 
\int^{\beta}_0 d\tau\,e^{i\epsilon_n \tau}
\frac{\partial}{\partial\tau} G_{\nu\alpha\nu^{\prime}\alpha^{\prime}}(\tau,\mathbf{k})
\end{eqnarray}
The above holds in the interacting case.  Specializing to
the non-interacting case we have
\begin{eqnarray}
G^{(0)}_{\nu\alpha\nu^{\prime}\alpha^{\prime}}(\epsilon_n,\mathbf{k}) & = &
\frac{\delta_{\alpha\alpha^{\prime}}\delta_{\nu\nu^{\prime}}}{i\varepsilon_n}
- \frac{1}{i\varepsilon_n} 
\int^{\beta}_0 d\tau\,e^{i\epsilon_n \tau}
\sum_{\nu^{\prime\prime}\alpha^{\prime\prime}} 
B_{\nu\alpha\nu^{\prime\prime}\alpha^{\prime\prime}}(\mathbf{k}) 
G^{(0)}_{\nu^{\prime\prime}\alpha^{\prime\prime}\nu^{\prime}\alpha^{\prime}}(\tau,\mathbf{k})\\
& = &
\frac{\delta_{\alpha\alpha^{\prime}}\delta_{\nu^{\prime}\nu}}{i\varepsilon_n}
- \frac{1}{i\varepsilon_n} \sum_{\nu^{\prime\prime}\alpha^{\prime\prime}}  
B_{\nu\alpha\nu^{\prime\prime}\alpha^{\prime\prime}}(\mathbf{k})
G^{(0)}_{\nu^{\prime\prime}\alpha^{\prime\prime}\nu^{\prime}\alpha^{\prime}}(\epsilon_n,\mathbf{k})
\end{eqnarray}
In matrix notation we have
\begin{eqnarray}
i\varepsilon_n\, G^{(0)} & = & I - B   G^{(0)} \\
(i\varepsilon_n I + B)\,G^{(0)} & = & I \\
(G^{(0)})^{-1} & = & (i\varepsilon_n I + B).
\end{eqnarray}
From this it is clear that $B$ is the proper generalization
of $-H^{(0)}$ for this represenation.
Focusing on the one-band case for concreteness, we have
\begin{equation}
H^{(0)}(\mathbf{k}) =
\begin{pmatrix}
\xi_{\mathbf{k}} - h_z &
- h_x + i h_y &
h_p\, \phi^*_{\uparrow\uparrow}(-\mathbf{k}) &
h_p\, \phi^*_{\uparrow\downarrow}(-\mathbf{k}) \\
- h_x - i h_y &
\xi_{\mathbf{k}} + h_z &
h_p\, \phi^*_{\downarrow\uparrow}(-\mathbf{k}) &
h_p\, \phi^*_{\downarrow\downarrow}(-\mathbf{k}) \\ 
h_p\, \phi_{\uparrow\uparrow}(-\mathbf{k}) &
h_p\, \phi_{\downarrow\uparrow}(-\mathbf{k}) &
- \xi_{-\mathbf{k}} + h_z &
h_x + ih_y \\
h_p\, \phi_{\uparrow\downarrow}(-\mathbf{k}) &
h_p\, \phi_{\downarrow\downarrow}(-\mathbf{k}) & 
h_x - ih_y &
- \xi_{-\mathbf{k}} - h_z 
\end{pmatrix}
\end{equation}

Let $z_1, z_2, z_3$ and $z_4$ be the
$\mathbf{k}$-dependent eigenvalues
for $H^{(0)}$ and let
$\hat{A}$ and $\hat{A}^{-1}$ be the
transformation matrices that convert this
matrix to diagonal form.  With this we have
\begin{eqnarray}
G^{(0)}(\varepsilon_n,\mathbf{k})^{-1} & = & 
\hat{A}
\begin{pmatrix}
i\varepsilon_n - z_1 & 0 & 0 & 0 \\
0 & i\varepsilon_n - z_2 & 0 & 0 \\
0 & 0 & i\varepsilon_n - z_3 & 0 \\
0 & 0 & 0 & i\varepsilon_n - z_4
\end{pmatrix}
\hat{A}^{-1} \\
G^{(0)}(\varepsilon_n,\mathbf{k}) & = & 
\hat{A} 
\begin{pmatrix}
1/ (i\varepsilon_n - z_1) & 0 & 0 & 0 \\
0 & 1/(i\varepsilon_n - z_2) & 0 & 0 \\
0 & 0 & 1/(i\varepsilon_n - z_3) & 0 \\
0 & 0 & 0 & 1/(i\varepsilon_n - z_4)
\end{pmatrix}
\hat{A}^{-1}
\end{eqnarray} 

Next we transform from the (imaginary) frequency, $\varepsilon_n$ to
the (imaginary) time domain, $\tau$.
By definition, this transformation is given
by
\begin{equation}
G^{(0)}(\tau,\mathbf{k}) \equiv T \sum_{\epsilon_n}
e^{-i \varepsilon_n \tau}G^{(0)}(\varepsilon_n,\mathbf{k}).
\end{equation}
As shown in the appendix, for $\tau > 0$,
\begin{equation}
 T \sum_{\epsilon_n}
e^{-i \varepsilon_n \tau} \frac{1}{i\varepsilon_n - z_i}
=  
 - \frac{e^{-z_i \tau}}
{e^{-z_i \beta} + 1}.
\end{equation}
Therefore,
\begin{equation}
G^{(0)}(\tau > 0,\mathbf{k}) = 
\hat{A} 
\begin{pmatrix}
- \frac{e^{-z_1 \tau}}
{e^{-z_1 \beta} + 1} & 0 & 0 & 0 \\
0 & - \frac{e^{-z_2 \tau}}
{e^{-z_2 \beta} + 1}
 & 0 & 0 \\
0 & 0 & - \frac{e^{-z_3 \tau}}
{e^{-z_3 \beta} + 1} & 0 \\
0 & 0 & 0 & - \frac{e^{-z_4 \tau}}
{e^{-z_4 \beta} + 1}
\end{pmatrix}
\hat{A}^{-1}
\end{equation}
This expression is used only for the special case
of $\tau \to 0^+$.  In that case we have
\begin{equation}
G^{(0)}(\tau \to 0^+,\mathbf{k}) = 
\hat{A} 
\begin{pmatrix}
- 
(e^{-z_1 \beta} + 1)^{-1} & 0 & 0 & 0 \\
0 & - (e^{-z_2 \beta} + 1)^{-1}  & 0 & 0 \\
0 & 0 & - (e^{-z_3 \beta} + 1)^{-1} & 0 \\
0 & 0 & 0 & - (e^{-z_4 \beta} + 1)^{-1}
\end{pmatrix}
\hat{A}^{-1}
\end{equation}

\section{Green's function and self-energy symmetries}

Since the field operators are not independent, some of 
the Green's functions are redundant.  Expressions that
involve traces must be carefully constructed since
tracing over all the components of the Green's function
effectively counts some terms once and others twice.

By definition we have
\begin{eqnarray}
G_{\nu 2 \nu^{\prime}2}(\tau ,\mathbf{r}; \tau^{\prime},\mathbf{r}^{\prime})
& = & -{\cal T}_{\tau} <c^{\dagger}_{\nu\uparrow}(\tau,\mathbf{r})
c_{\nu^{\prime}\uparrow}(\tau^{\prime},\mathbf{r}^{\prime})> \\
G_{\nu^{\prime}0\nu 0}(\tau ,\mathbf{r}; \tau^{\prime},\mathbf{r}^{\prime})
& = & -{\cal T}_{\tau} <c_{\nu^{\prime}\uparrow}(\tau,\mathbf{r})
c^{\dagger}_{\nu\uparrow}(\tau^{\prime},\mathbf{r}^{\prime})>.
\end{eqnarray}
Thus, for $\tau > \tau^{\prime}$ we have
\begin{eqnarray}
G_{\nu 2 \nu^{\prime}2}(\tau ,\mathbf{r}; \tau^{\prime},\mathbf{r}^{\prime})
& = & -<c^{\dagger}_{\nu\uparrow}(\tau,\mathbf{r})
c_{\nu^{\prime}\uparrow}(\tau^{\prime},\mathbf{r}^{\prime})> \\
G_{\nu^{\prime}0 \nu 0}(\tau^{\prime},\mathbf{r}^{\prime};\tau ,\mathbf{r})
& = & <c^{\dagger}_{\nu \uparrow}(\tau,\mathbf{r})
c_{\nu^{\prime}\uparrow}(\tau^{\prime},\mathbf{r}^{\prime})>
\end{eqnarray}
from which it is clear that
\begin{eqnarray}
G_{\nu 2 \nu^{\prime}2}(\tau ,\mathbf{r}; \tau^{\prime},\mathbf{r}^{\prime}) & = &
- G_{\nu^{\prime}0\nu 0}(\tau^{\prime},\mathbf{r}^{\prime};\tau ,\mathbf{r}) \\
G_{\nu 3 \nu^{\prime}3}(\tau ,\mathbf{r}; \tau^{\prime},\mathbf{r}^{\prime}) & = &
- G_{\nu^{\prime}1\nu 1}(\tau^{\prime},\mathbf{r}^{\prime};\tau ,\mathbf{r})
\end{eqnarray}
where the last identity follows by replacing up spins with down spins.

Some of the off-digonal terms are related by symmetry as well.
By defintion
\begin{eqnarray}
G_{\nu 0 \nu^{\prime}3}(\tau,\mathbf{r};\tau^{\prime}, \mathbf{r}^{\prime}) & = &
-{\cal T}_{\tau} <c_{\nu\uparrow}(\tau,\mathbf{r})\,
c_{\nu^{\prime}\downarrow}(\tau^{\prime},\mathbf{r}^{\prime})> \\
G_{\nu^{\prime}1\nu 2}(\tau,\mathbf{r}; \tau^{\prime},\mathbf{r}^{\prime}) & = &
-{\cal T}_{\tau} <c_{\nu^{\prime}\downarrow}(\tau,\mathbf{r})\,
c_{\nu \uparrow}(\tau^{\prime},\mathbf{r}^{\prime})>.
\end{eqnarray}
For $\tau > \tau^{\prime}$ we get
\begin{eqnarray}
G_{\nu 0 \nu^{\prime}3}(\tau,\mathbf{r}; \tau^{\prime},\mathbf{r}^{\prime}) & = &
- <c_{\nu\uparrow}(\tau,\mathbf{r})\,
c_{\nu^{\prime}\downarrow}(\tau^{\prime},\mathbf{r}^{\prime})> \\
G_{\nu^{\prime}1\nu 2}(\tau^{\prime},\mathbf{r}^{\prime};\tau, \mathbf{r}) & = &
<c_{\nu\uparrow}(\tau,\mathbf{r})\,
c_{\nu^{\prime}\downarrow}(\tau^{\prime},\mathbf{r}^{\prime})>.
\end{eqnarray}
A similar proof can be used for the creation operator paired
terms to get
\begin{eqnarray}
G_{\nu 1 \nu^{\prime}2}(\tau,\mathbf{r}; \tau^{\prime},\mathbf{r}^{\prime}) & = &
- G_{\nu^{\prime}0\nu 3}(\tau^{\prime},\mathbf{r}^{\prime}; \tau,\mathbf{r}) \\
G_{\nu 2 \nu^{\prime}1}(\tau,\mathbf{r};\tau^{\prime}, \mathbf{r}^{\prime}) & = &
- G_{\nu^{\prime}3\nu 0}(\tau^{\prime},\mathbf{r}^{\prime};\tau, \mathbf{r})
\end{eqnarray}

Similar symmetries hold for the self-energy.
By definition we have
\begin{equation}
\Sigma_{\nu\alpha\nu^{\prime}\alpha^{\prime}}(x,x^{\prime}) = T 
\frac{\delta \Phi}{\delta G_{\nu^{\prime}\alpha^{\prime}\nu\alpha}(x^{\prime},x)}.
\end{equation}
Therefore
\begin{eqnarray}
\Sigma_{\nu 2 \nu^{\prime}2}(x,x^{\prime}) & = & T 
\frac{\delta \Phi}{\delta G_{\nu^{\prime}2\nu 2}(x^{\prime},x)} \\
& = & -T 
\frac{\delta \Phi}{\delta G_{\nu 0 \nu^{\prime}0}(x,x^{\prime})} \\
& = & -\Sigma_{\nu^{\prime}0\nu 0}(x^{\prime},x)
\end{eqnarray}
Likewise $\Sigma_{\nu 3 \nu^{\prime}3}(x,x^{\prime}) = 
-\Sigma_{\nu^{\prime} 1 \nu 1}(x^{\prime},x)$.

Similar symmetries hold for off-diagonal components.
\begin{eqnarray}
\Sigma_{\nu 1 \nu^{\prime}2}(x,x^{\prime}) & = & T \frac{\delta \Phi}
{\delta G_{\nu^{\prime}2 \nu 1}(x^{\prime},x)} \\
& = & -T \frac{\delta \Phi}{\delta G_{\nu 3 \nu^{\prime}0}(x,x^{\prime})} \\
& = & -\Sigma_{\nu^{\prime} 0 \nu 3}(x^{\prime},x).
\end{eqnarray}
Likewise $\Sigma_{\nu 2 \nu^{\prime}1}(x,x^{\prime})  
=  -\Sigma_{\nu^{\prime}3 \nu 0}(x^{\prime},x)$.

These symmetry relations can be summarized with the rule
\begin{equation}
G_{\nu\alpha\nu^{\prime}\alpha^{\prime}}(x,x^{\prime}) = 
-G_{\nu^{\prime}\overline{\alpha^{\prime}}\nu\overline{\alpha}}(x^{\prime},x)
\end{equation}
where $\overline{\alpha}$ is the antiparticle to
particle $\alpha$, \textit{i.e.} if $\alpha = 0$, then
$\overline{\alpha} = 2$.  This suggests that one of
the symmetries missed above is
\begin{equation}
G_{\nu 0 \nu^{\prime}2}(x,x^{\prime}) = -G_{\nu^{\prime}0 \nu 2}(x^{\prime},x).
\end{equation}
In the limit $\tau = \tau^{\prime}$ the above reiterates
that the spatial part of the pair-wave function for spin
up electrons must be antisymmetric in $\mathbf{r}$.
