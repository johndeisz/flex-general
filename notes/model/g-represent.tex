\chapter{Mixed analytical/numerical representation of the Green's function}
\label{chapter:grepresent}

Consider the following representation for the 
Green's function,
\begin{equation}
G(\varepsilon_n, \mathbf{r}) \equiv g(\varepsilon_n,\mathbf{r})
+ \tilde{G}(\varepsilon_n,\mathbf{r})
\end{equation}
where
\begin{equation}
\label{little_g}
\begin{split}
g(\varepsilon_n,\mathbf{r})  = & -( \Delta G(\mathbf{r})  - c_0(\mathbf{r}) )
Q_0(x_{00},\varepsilon_n) - c_0(\mathbf{r}) Q_0(x_{01}, \varepsilon_n) \\
& + (\Delta G^{\prime}(\mathbf{r}) - c_1(\mathbf{r}))
Q_1(x_{10},\varepsilon_n) + c_1(\mathbf{r})Q_1(x_{11},\varepsilon_n)
\end{split}
\end{equation}
In this case, $g$, $\Delta G$, $\Delta G^{\prime}$, $c_0$, and
$c_1$ are understood to be $2 \times 2$ matrices for each
value of $\mathbf{r}$.
Since $Q_0$ and $Q_1$ have special asymptotic properties
(as described in an appendix), the above functional
form for $g(\varepsilon_n,\mathbf{r})$ guarantees
that it contains the full $1/(i \varepsilon_n)$ and
$1/(i \varepsilon)^2$ contributions
to $G(\varepsilon_n, \mathbf{r})$
and $|\varepsilon_n| \to \infty$.

The $\mathbf{r}$-dependent 
numerical parameters $c_0(\mathbf{r})$ and $c_1(\mathbf{r})$ are used to
satisfy the following:
\begin{eqnarray}
\tilde{G}(\tau \to 0^+, \mathbf{r}) & = & 0 \\
\tilde{G}^{\prime}(\tau \to 0^+, \mathbf{r}) & = & 0.
\end{eqnarray}
These conditions do not
constrain the values 
for $x_{00}$, $x_{01}$, $x_{10}$, and $x_{11}$.
Thus, these parameters will be taken as fixed and independent
of $\mathbf{r}$.  Note that this is different from the
algorithm developed earlier.

To determine the values of $c_0$ and $c_1$ we
set $G(\tau \to 0^+, \mathbf{r}) = g(\tau \to 0^+, \mathbf{r})$
and $G^{\prime}(\tau \to 0^+, \mathbf{r}) = 
g^{\prime}(\tau \to 0^+, \mathbf{r})$.
The first condition gives
\begin{equation}
\begin{split}
G(\tau \to 0^+, \mathbf{r}) = &
- (\Delta G(\mathbf{r}) - c_0(\mathbf{r})) \,Q_0(x_{00},\tau \to 0^+)
 - c_0(\mathbf{r}) Q_0(x_{01}, \tau \to 0^+) \\
& +  (\Delta G^{\prime}(\mathbf{r}) - c_1(\mathbf{r})) \,Q_1(x_{10},\tau \to 0^+)
+ c_1(\mathbf{r}) \,Q_1(x_{11},\tau \to 0^+).
\end{split}
\end{equation}
Since
\begin{eqnarray}
Q_0(x_{0\alpha},\tau \to 0^+) & = & -\,\frac{1}{2} \\
Q_1(x_{1\alpha},\tau \to 0^+) & = & -\,\frac{1}{2 x_{1\alpha}}
\tanh(\beta x_{1\alpha}/2)
\end{eqnarray}
we get
\begin{equation}
\begin{split}
G(\tau \to 0^+, \mathbf{r}) = &
          - (\Delta G(\mathbf{r}) - c_0(\mathbf{r}))(-\frac{1}{2}) -
       c_0(\mathbf{r})(-\frac{1}{2}) \\
& + (\Delta G^{\prime}(\mathbf{r}) - c_1(\mathbf{r}))
\frac{- \tanh(\beta x_{10} / 2)}{2 x_{10}}
+ c_1(\mathbf{r}) \frac{- \tanh(\beta x_{11} / 2)}{2 x_{11}}
\end{split}
\end{equation}
After simplification we get
\begin{equation}
\begin{split}
G(\tau \to 0^+, \mathbf{r}) = &
\frac{\Delta G(\mathbf{r})}{2}  - \frac{\Delta G^{\prime}(\mathbf{r})
\,\tanh(\beta x_{10}/2)}{2 x_{10}} \\
& + c_1(\mathbf{r}) \left[ \frac{\tanh(\beta x_{10}/2)}{2 x_{10}}
 - \frac{\tanh(\beta x_{11}/2)}{2 x_{11}} \right]
\end{split}
\end{equation}
after which it is simple to solve for $c_1(\mathbf{r})$.   We get
\begin{equation}
c_1(\mathbf{r}) = \frac{G(\tau \to 0^+, \mathbf{r}) - 
\frac{\Delta G(\mathbf{r})}{2}
+ \frac{\Delta G^{\prime}(\mathbf{r})
\,\tanh(\beta x_{10}/2)}{2 x_{10}}}{
\left[\frac{\tanh(\beta x_{10}/2)}{2 x_{10}}
 - \frac{\tanh(\beta x_{11}/2)}{2 x_{11}} \right]}.
\end{equation} 

The second condition gives us
\begin{equation}
\label{g_prime}
\begin{split}
G^{\prime}(\tau \to 0^+, \mathbf{r}) = &
- (\Delta G(\mathbf{r}) - c_0(\mathbf{r})) 
\,Q_0^{\prime}(x_{00},\tau \to 0^+)
 - c_0(\mathbf{r}) Q_0^{\prime}(x_{01}, \tau \to 0^+) \\
& +  (\Delta G^{\prime}(\mathbf{r}) - c_1(\mathbf{r})) 
\,Q_1^{\prime}(x_{10},\tau \to 0^+)
+ c_1(\mathbf{r}) \,Q_1^{\prime}(x_{11},\tau \to 0^+).
\end{split}
\end{equation}
Since
\begin{eqnarray}
Q_0^{\prime}(x_{0\alpha},\tau \to 0^+) & = & \frac{x_{0\alpha}}{2}
\tanh(\beta x_{0\alpha}/2) \\
Q_1^{\prime}(x_{1\alpha},\tau \to 0^+) & = & \frac{1}{2}
\end{eqnarray}
Eq.~(\ref{g_prime}) becomes
\begin{equation}
\begin{split}
G^{\prime}(\tau \to 0^+, \mathbf{r}) = &
- (\Delta G(\mathbf{r}) - c_0(\mathbf{r})) \frac{x_{00}}{2} 
\tanh(\beta x_{00}/2) - c_0(\mathbf{r}) \frac{x_{01}}{2} 
\tanh(\beta x_{01}/2) \\
& +  (\Delta G^{\prime}(\mathbf{r}) - c_1(\mathbf{r}))\frac{1}{2}
+ c_1(\mathbf{r})\frac{1}{2}.
\end{split}
\end{equation}
After simplification we get
\begin{equation}
\begin{split}
G^{\prime}(\tau \to 0^+, \mathbf{r}) = &
\frac{1}{2}\Delta G^{\prime}(\mathbf{r}) -
 \Delta G(\mathbf{r}) \frac{x_{00}}{2} 
\tanh(\beta x_{00}/2) \\
& + c_0(\mathbf{r}) \left(\frac{x_{00}}{2} \tanh(\beta x_{00}/2)
- \frac{x_{01}}{2} \tanh(\beta x_{01}/2) \right).
\end{split}
\end{equation}
Finally, we solve for $c_0(\mathbf{r})$ to get
\begin{equation}
c_0(\mathbf{r}) = \frac{ G^{\prime}(\tau \to 0^+, \mathbf{r}) -
\frac{1}{2}\Delta G^{\prime}(\mathbf{r}) +
 \Delta G(\mathbf{r}) \frac{x_{00}}{2} 
\tanh(\beta x_{00}/2) }
{\left(\frac{x_{00}}{2} \tanh(\beta x_{00}/2)
- \frac{x_{01}}{2} \tanh(\beta x_{01}/2) \right)}.
\end{equation}

It is convenient to rewrite Eq.~(\ref{little_g})
in the following mathematically equivalent form:
\begin{equation}
g(\tau, \mathbf{r}) = \sum_{\alpha = 0,1; \beta = 0, 1}
c_{\alpha\beta}(\mathbf{r}) Q_{\alpha\beta}(\tau).
\end{equation}
Here we set
\begin{eqnarray}
Q_{00}(\tau) & \equiv & Q_0(x_{00},\tau) \\
Q_{01}(\tau) & \equiv & Q_0(x_{00},\tau) \\
Q_{10}(\tau) & \equiv & Q_1(x_{10},\tau) \\
Q_{11}(\tau) & \equiv & Q_1(x_{11},\tau). 
\end{eqnarray}
In this modified representation we have
\begin{eqnarray}
c_{01}(\mathbf{r}) & = & - \frac{ G^{\prime}(\tau \to 0^+, \mathbf{r}) -
\frac{1}{2}\Delta G^{\prime}(\mathbf{r}) +
 \Delta G(\mathbf{r}) \frac{x_{00}}{2} 
\tanh(\beta x_{00}/2) }
{\left(\frac{x_{00}}{2} \tanh(\beta x_{00}/2)
- \frac{x_{01}}{2} \tanh(\beta x_{01}/2) \right)} \\
c_{00}(\mathbf{r}) & = & - \Delta G(\mathbf{r}) - 
c_{01}(\mathbf{r}) \\
c_{11}(\mathbf{r}) & = & \frac{G(\tau \to 0^+, \mathbf{r}) - 
\frac{\Delta G(\mathbf{r})}{2}
+ \frac{\Delta G^{\prime}(\mathbf{r})
\,\tanh(\beta x_{10}/2)}{2 x_{10}}}{
\left[\frac{\tanh(\beta x_{10}/2)}{2 x_{10}}
 - \frac{\tanh(\beta x_{11}/2)}{2 x_{11}} \right]} \\
c_{10}(\mathbf{r}) & = & \Delta G^{\prime}(\mathbf{r}) - c_{11}(\mathbf{r}).
\end{eqnarray}

On the first iteration (where the values of $G(\tau \to 0^{+},\mathbf{r})$
and $G^{\prime}(\tau \to 0^+,\mathbf{r})$ are unknown), we set
$c_{01}(\mathbf{r}) = 0$, $c_{00}(\mathbf{r}) = - \Delta G(\mathbf{r})$,
$c_{11}(\mathbf{r}) = 0$, and $c_{10}(\mathbf{r}) =
\Delta G^{\prime}(\mathbf{r})$.
This choice guarantees, at least, that $g(\tau,\mathbf{r})$ will
have the correct asymptotic properties.


